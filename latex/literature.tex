%\section*{Литература}
%\addcontentsline{toc}{section}{Литература}

%\vspace{2em}

%\begingroup
%\renewcommand{\section}[2]{}%

\bibliographystyle{gost780u}
\bibliography{literature}

%\begin{thebibliography}{}
    %\bibitem{Chebotarev}{\sc А.Ю. Чеботарёв}, “Конечномерная
        %стабилизация с заданной скоростью систем типа Навье – Стокса”,
        %Дальневост. матем. журн., 10:2 (2010), 199–204 

    %\bibitem{ChebotarevMGT}{\sc А.Ю. Чеботарёв}, “Стабилизация
        %сторонними токами равновесных МГД конфигураций”, Ж. вычисл. матем. и
        %матем. физ., 52:12 (2012), 2238–2246

    %\bibitem{ChebotarevBS}{\sc А.Ю. Чеботарёв}, “Конечномерная стабилизация
        %уравнения Бюргерса” (в печати)

    %\bibitem{KMV}{\sc Miroslav Krstic, Lionel Magnis, Rafael Vazquez} Nonlinear 
        %Control of the Burgers PDE—Part I: Full-State Stabilization.\ June 11-13, 2008

    %\bibitem{KS}{\sc M. Krstic, A. Smyshlyaev} Boundary Control of PDEs.\ 2008

    %\bibitem{KWCK}{\sc A. Kwiecinska} Stabilization of partial differential
        %equations by noise, Stochastic Processes and their Applications 79
        %(1999) 179–184

    %\bibitem{Barbu}{\sc Barbu V.} Feedback stabilization of Navier – Stokes
        %equations// ESAIM: Control, Optimisation and Calculus of Variations. 
        %2003. V.9. P. 197-205

    %\bibitem{}{\sc A. Kourbatov} Lyapunov Exponents for Burgers’ Equation. 
        %Sbornik Nauchnykh Trudov MFTI. Moscow, 1992

    %\bibitem{Furs}{\sc А. В. Фурсиков}, “Стабилизируемость квазилинейного
            %параболического уравнения с помощью граничного управления с обратной
            %связью”, Матем. сб., 192:4 (2001), 115–160

    %\bibitem{} Краевые задачи математической физики и смежные вопросы теории
        %функций. 31, Зап. научн. сем. ПОМИ, 271, ред. О. А. Ладыженская, ПОМИ,
        %СПб., 2000

    %\bibitem{Samar}{\sc Самарский А.А., Гулин А.В.}, Численные методы.
            %-М.:Наука,1989.-432с.

    %\bibitem{Philon} Формула Филона [Электронный ресурс] : 
        %\url{http://window.edu.ru/resource/886/19886/files/rsu177.pdf}

    %\bibitem{TDMA} Метод прогонки, неявная схема [Электронный ресурс] : 
        %\url{http://ikt.muctr.ru/html2/4/lek4_2_1.html}

    %\bibitem{}{\sc Barbu V., Triggiani R.} Internal stabilization of Navier – Stokes 
        %equations with finite – dimensional controllers// Indiana Univ. 
        %Math. J. 2004. V.53. N. 5. P. 1443-1494.

    %\bibitem{Agrachev A.A., Sarychev A.V.} Navier – Stokes equations: 
        %controllability by means of low modes forcing// J. Math. Fluid Mech. 
        %2006. V.7. P. 108-152.

    %\bibitem{Barbu V.} Analysis and control of nonlinear infinite dimensional systems.
        %Academic Press, New York, 1993.

%\end{thebibliography}

%\endgroup
