%\documentclass[a5paper,14pt]{G7-32}
\documentclass[utf8x]{G7-32}
%\usepackage{extsizes}
%\usepackage[T2A]{fontenc}
%\usepackage[utf8x]{inputenc}
%\usepackage[english,russian]{babel}
%\usepackage{amssymb,amsfonts,amsmath,mathtext,cite,enumerate,float}
%% \usepackage[dvips]{graphicx}
%%\usepackage{breqn}
\usepackage{graphicx}
%%\usepackage{sectsty}
%\usepackage{array}
\usepackage{amsthm}
%%\usepackage{titlesec}
%\usepackage{indentfirst}
%\usepackage{xr}
%\usepackage{subfigure}
%\usepackage[round, sort, numbers]{natbib}
\usepackage{url}
%\usepackage{setspace}
\usepackage{pdfpages}
%\usepackage{caption}
%\usepackage{glossaries}
\usepackage{float}

%\setcitestyle{square}
\graphicspath{{images/}}

%\allsectionsfont{\centering}

%\makeatletter
% \renewcommand{\@biblabel}[1]{#1.}
%\makeatother

%\usepackage{geometry}
%\geometry{left=2.5cm}
%\geometry{right=1.5cm}
%\geometry{top=2cm}
%\geometry{bottom=2cm}

\include{latex-g7-32/tex/preamble.inc}

% Настройки листингов.
\ifPDFTeX
\include{latex-g7-32/tex/plistings.inc}
\else
\usepackage{latex-g7-32/tex/plocal-minted}
\fi

\include{latex-g7-32/tex/macros.inc}

\newcommand{\norm}[1]{\left\lVert#1\right\rVert}
\newcommand{\isum}[2][j]{\sum \limits_{#1=1}^{\infty}{#2}}
\newcommand{\operator}[1]{\mathcal{R}{#1}}
\newcommand{\supp}{\mathop{\mathrm{supp}}}
%\newtheorem{remark}{Замечание}

\DeclareMathOperator\arctanh{arctanh}
\DeclareMathOperator\arccoth{arccoth}
\DeclareMathOperator\arccot{arccot}

\numberwithin{equation}{chapter}

%\newcommand{\sectionbreak}{\clearpage}

\begin{document}

%\includepdf[pages={1-}]{titul.pdf}
%\renewcommand{\contentsname}{Содержание}
%\clearpage
%\setcounter{page}{1}
%\tableofcontents

%\onehalfspacing
%\begin{onehalfspace}
    %%\chapter*{Введение}
%\addcontentsline{toc}{section}{Введение}
%\vspace{1em}

\Introduction

В реальных физических процессах неизбежно возникают непредусмотренные 
флуктуации, и поэтому возникает необходимость разработки методов построения 
управлений, способных реагировать на непредусмотренные возмущения и подавлять 
их \cite{Furs}. Проблемы стабилизации систем параболического типа привлекают внимание 
специалистов в силу прикладной значимости данной системы \cite{Chebotarev}. 
Управления такого типа называются \emph{управлениями с обратной связью} \cite{KS}.

Вопрос о стабилизируемости различных эволюционных уравнений в частных 
производных с помощью управлений исследовался многими авторами, среди которых 
A.Kwiecinka \cite{KWCK}, Barbu V. \cite{Barbu}, M. Kristic
\cite{KMV, KS}, Фурсиков А.В. \cite{Furs}, Чеботарёв А.Ю. 
\cite{Chebotarev, ChebotarevBS, ChebotarevMGT} и разработаны методы управления, 
такие как : стабилизация Ляпунова, метод backstepping, управление с обратной
связью. Существующие методы стабилизации довольно громоздкие и требуют значительных 
вычислительных затрат, тогда как алгоритм рассматриваемый в ВКР прост в реализации.

Представленная ВКР посвящена изучению алгоритмам стабилизации для неустойчивых
по Ляпунову линейных и нелинейных параболических систем. В работе
рассматривается алгоритм стабилизация эволюционных систем с постоянными коэффициентами 
конечномерным локальным управлением с обратной связью. Указанный
метод позволяет провести стабилизацию с любой наперёд заданной скоростью $\sigma
> 0$ засчет выбора параметров стабализирующего оператора.

В работе проделаны теоретический анализ устойчивости линейной параболической
системы на примере уравнения теплопроводности, численные примеры иллюстрирующие
неустойчивые решения, примеры численного моделирования стабилизации, теоретическое
обоснование алгоритма стабилизации, предложен метод стабилизации неустойчивых
стационарных решений типа "shock-like" для вязкого уравнения Бюргерса.

    %%%\section*{Основные обозначения}
%\addcontentsline{toc}{section}{Основные обозначения}
%\vspace{3em}

\Abbreviations

\begin{description}

    \item{Символы $u_t, u_x, u_{xx}$ обозначают соотвествующие классические
        производные функции $u$}

    \item{Через $E'$ обозначим пространство сопряженное к пространству $E$}

    \item{$L^{p}(\Omega), \ p \ge 1,$ - банахово пространство (т.е. полное линейное
нормированное пространство), состоящее из всех определенных и измеримых (по
Лебегу) на $\Omega$ функций, имеющих конечную норму
\begin{equation*}
    \norm{u}_{p, \Omega} = (\int\limits_{\Omega}{|u|^p dx})^{1/p}
\end{equation*}
Норму в $L^2(\Omega)$ обозначим коротко $\norm{\ .\ }$, а скалярное произведение
как $(\cdot,\cdot)$. Этим же символом будем обозначать отношения двойственности
между $E$ и $E'$.}\cite{OAL}

\item{$W^l_m(\Omega)$(Пространство Соболева) - функциониональное пространство,
состоящее из функций из пространства Лебега, имеющих обобщенные производные 
заданного порядка из $L^m(\Omega)$. В частности
$W^1_2(\Omega)$ состоит из элементов $L^2(\Omega)$, имеющих квадратично
суммируемые по $\Omega$ обобщенные производные первого порядка. В дальнейшим,
через $H^l(\Omega)$ обозначим $W^l_2(\Omega)$.}

\item{Здесь и далее, $\Omega = (0, 1) \subset \mathbb{R}$, $H = L^2(\Omega)$ и 
    $V = H^1_0(\Omega)$}

\end{description}



    %%\section{Стабилизация неустойчивых параболических систем}

\subsection{Анализ устойчивости линейного параболического уравнения}

Рассмотрим параболическое уравнение

\begin{equation}\label{dif_form}
    u_t = u_{xx} + \alpha u, \ 0 < x < 1, \ t > 0
\end{equation}

с начальным и граничными условиями:

\begin{gather}\label{d_control}
    u(0, t) = u(1, t) = 0, \\*
    u(x, 0) = u_{0}(x) \in L^2(0, 1). \nonumber
\end{gather}

Здесь и далее через $u_t$, $u_x$, $u_{xx}$ .. обозначаются соотвествующие
частные производные функции $u$.\\
Умножим уравнение \eqref{dif_form} на $u$ скалярно в $L^2(0, 1)$

\begin{equation*}
    (u_t, u) = (u_{xx}, u) + \alpha (u, u).
\end{equation*}

Скалярное произведение в $L^2(0, 1)$ определяется как $(u, v) = \int_0^1 uv dx$,
а норма как $\norm{u} = \sqrt{(u, u)}$. Получаем

\begin{equation}\label{int_form}
    \frac{1}{2}\frac{d}{dt}\norm{u}^2 = -\norm{u_x}^2 + \alpha \norm{u}^2.
\end{equation}

C помощью неравенства Пуанкаре–Фридрихса-Стеклова

\begin{equation*}
    \norm{u}^2 \le \frac{1}{\pi^2} \norm{u_x}^2.
\end{equation*}
получаем следующую оценку 
\begin{equation}\label{stable_opr}
    \frac{1}{2}\frac{d}{dt}\norm{u}^2 \le (\alpha - \pi^2)\norm{u}^2.
\end{equation}
Рассмотрим 3 случая

\begin{enumerate}
    \item $\alpha = \pi^2$. Тогда из \eqref{stable_opr} следует неравенство
        \begin{equation}
            \norm{u}^2 \le \norm{u_0}^2.
        \end{equation}
        
        Указанное неравенство означает, что нулевое решение задачи
        \eqref{dif_form} - \eqref{d_control} устойчиво по Ляпунову, но не устойчиво ассимптотически
    
    \item $\alpha < \pi^2$. Обозначим $\frac{\mu}{2} = -(\alpha - \pi^2)$.\\
        
        Тогда

        \begin{equation}\label{less_pi2}
            \frac{d}{dt}\norm{u}^2 + \mu \norm{u}^2 \le 0
        \end{equation}

        Домножим обе части \eqref{less_pi2} на $e^{\mu t}$. Тогда, 
        $\frac{d(\norm{u}^2 e^{\mu t})}{dt} \le 0$. Проинтегрируем и в итоге получим

        \begin{equation*}
            \norm{u}^2 \le \norm{u_0}^2 e^{-\mu t} = \norm{u_0}^2 e^{2(\alpha - \pi^2) t}
        \end{equation*}

        Полученная оценка гарантирует ассимптитическую экспоненциальную устойчивость.

    \item $\alpha > \pi^2$. Решение начально краевой задачи \eqref{dif_form} - \eqref{d_control} имеет вид
        \begin{equation}
            u(x, t) = 2 \isum{a_j e^{(\alpha - \pi^2 j^2)t}\sin{(\pi j x)}}
        \end{equation}

        Здесь $a_j = \int_0^1{u_0 \sin{(\pi j s)} ds}$. Первый член суммы указывает на темп роста решения при $t \rightarrow \infty$. Следовательно, система неустойчива.
\end{enumerate}

Для стабилизации системы \eqref{int_form} в случае 3, будем использовать ниже описанный метод.

    %%\subsection{Численные примеры неустойчивых решений уравнения теплопроводности}
\vspace{1em}

\newtheorem{exmp}{Пример}

\begin{exmp}
\end{exmp}

В качестве начальных условий выберем $u_0 = \sin(\pi x)$. Рис. 1. иллюстрирует 
неустойчивость нулевого решения при $\alpha = \pi^2 + 0.1$.

\begin{figure}[H]
    \centering
        \includegraphics[width=3.5in]{par_ex_pi01}
        \caption{}
        \label{fig:test1}
\end{figure}

\begin{exmp}
\end{exmp}

На рис. 2 приведен график решения задачи \eqref{dif_form} - \eqref{d_control}
при $\alpha = \pi^2 + 3$ и $u_0(x) = x(1 - x)$, который демонстрирует 
экспоненциальный рост решения при увеличении $t$.

\begin{figure}[H]
    \centering
        \includegraphics[width=3.5in]{par_ex_pi3}
        \caption{}
        \label{fig:test1}
\end{figure}


    %%%\section{Стабилизация конечномерным локальным управлением с обратной связью}
%\vspace{1em}
\section{Стабилизация конечномерным локальным управлением с обратной связью}

Рассмотрим систему

\begin{equation}\label{ndif_form}
    u_t = u_{xx} + \alpha u, \ x \in \Omega, \ t > 0.
\end{equation}
с начальным и граничными условиями
\begin{gather}
    u(0, t) = u(1, t) = 0, \\*
    u(x, 0) = u_{0}(x) \in H .\nonumber
\end{gather}

Как показано в \S 1, в случае, когда $\alpha > \pi^2$, нулевое решение уравнения 
\eqref{ndif_form} неустойчиво.
Задача стабилизации параболического уравнения, заданного в ограниченной области,
заключается в построении такого оператора управления, чтобы решение смешанной 
краевой задачи стремилось (при $t \rightarrow \infty$) к заданному стационарному 
решению с предписанной скоростью $e^{(-\delta_0t)}$.

Сформулируем задачу стабилизации неустойчивого нулевого решения уравнения 
\eqref{ndif_form} за счет локального управления.\\

Пусть $\omega \subset \Omega$ - произвольный интервал такой, что 
$\bar{\omega} \subset \Omega$.Задача стабилизации за счет конечномерных локально 
распределённых в $\omega$ управлений заключается в построении оператора 
$\mathcal{R} : H \rightarrow H$ такого, что

\begin{enumerate}
    \item $\forall z \in H \ \mathbf{supp} \ \operator{z} \subset \omega$,
    \item $\dim \operator{(H)} < +\infty$.
\end{enumerate}
и при этом решение задачи
\begin{gather}
    u_t - u_{xx} - \alpha u = \operator{u}. \nonumber\\
    u(0, t) = u(1, t) = 0, \ u(x, 0) = u_0. \nonumber
\end{gather}
экспоненциально стремится к нулю при $t \rightarrow + \infty$.

\section{Конструкция оператора управления}
\vspace{1em}

Заметим, что функции

\begin{equation}\label{basis}
    w_j = w_j(x) = \sqrt{2}\sin{(\pi j x)}, \ x \in \Omega, \ j=1, 2, ..
\end{equation}
образуют базис в $H$ и в $V$, причем в $H$ базиc ортонормирован.\\
Через $H_m$ обозначим подпространство в $H$, образованное первыми $m$ функциями 
из \eqref{basis}.\\

Далее рассмотрим следующие операторы проектирования

$$P_m : H \rightarrow H_m, \ Q_m : H \rightarrow H_m^{\perp}.$$

\begin{equation}
    P_m u = \sum \limits_{j=1}^{m} {(u, w_j) w_j}.
\end{equation}

\begin{equation}
    Q_m u = (I - P_m)u(x) = \sum \limits_{j=m + 1}^{\infty} {(u, w_j) w_j}.
\end{equation}

В качестве оператора стабилизиции будем рассматривать следующий конечномерный
оператор

$$\operator{z} = -r\chi_{\omega}P_mz, \ r > 0,$$
здесь
\begin{gather*}
    \begin{matrix}
        \chi_{\omega}(x) & =
        & \left\{
        \begin{matrix}
            0, & \mbox{если } x \notin \omega, \\
            1, & \mbox{иначе. }
        \end{matrix} \right.
    \end{matrix}
\end{gather*}

В следующем параграфе будет доказано, что существуют подходящие параметры 
$m \in \mathbb{N}$, $r = r_m > 0$, при которых $\operator{}$ обеспечивает 
стабилизацию неустойчивого решения.

\section{Теоретическое обоснование стабилизации}
\vspace{1em}

Рассмотрим уравнение
\begin{gather}\label{control}
    u_t - u_{xx} - \alpha u = -r\chi_{\omega}\varphi,\\*
    u|_{x = 0;1} = 0.
\end{gather}
Здесь $\varphi = P_mu$. Домножим скалярно обе части \eqref{control} на
$\varphi$. Учтем, что

\begin{gather*}
    (u, \varphi) = \norm{\varphi}^2,\\*
    (u_t, \varphi) = (\varphi_t, \varphi) = \frac{1}{2} \frac{d}{dt}
    \norm{\varphi}^2, \\*
    (u_{xx}, \varphi) = (\varphi_{xx}, \varphi) = -(\varphi_x, \varphi_x) = -
    \norm{\varphi_x}^2, \\*
    (\chi_{\omega}\varphi, \varphi) = \norm{\varphi}^2_{\omega}.
\end{gather*}
Тогда
\begin{equation*}
    \frac{1}{2} \frac{d}{dt} \norm{\varphi}^2  + \norm{\varphi_x}^2 - 
    \alpha \norm{\varphi}^2 + r \norm{\varphi}^2_{\omega} = 0.
\end{equation*}
Воспользуемся неравенством Пуанкаре-Фридрихса-Стеклова

\begin{equation}
    \norm{\varphi}^2 \le \frac{1}{\pi^2} \norm{\varphi_x}^2.
\end{equation}
В итоге получаем

\begin{equation}
    \frac{1}{2} \frac{d}{dt} \norm{\varphi}^2 + \pi^2 \norm{\varphi}^2 - 
    \alpha \norm{\varphi}^2 + r \norm{\varphi}^2_{\omega} \le 0.
\end{equation}

Приведём полезные для доказательства стабилизируемости леммы.

\newtheorem{lemma}{Лемма}

\begin{lemma}\label{util_lemma}
    Система $\left\{ w_j|_{\omega} \right\}^m_1$ линейно независима.
\end{lemma}

\begin{proof}
    % Пусть $D$ - оператор дифференцирования. Он является линейным отображением(преобразованием) $\mathbb{R}^m$ в $\mathbb{R}^m$. Рассмотрим следующее равенство :\\
    Cистема $\left\{ w_j|_{\omega} \right\}^m_1$ линейно независима, если 
    тождество вида
    
    \begin{equation}\label{sum_func}
        \sum \limits_{j = 1}^m{c_j w_j(x)} = 0, \ x \in \omega.
    \end{equation}
    выполняется только при $c_1 = c_2 = ... = c_m = 0.$\\

    Пусть $D$ - оператор дифференцирования, действующий в пространстве бесконечно 
    дифференцируемых функций.
    Заметим, что $D^2 w_j = -(\pi k)^2 w_j$. Подействуем  на \eqref{sum_func} 
    оператором $D^{2l}$ раз

    \begin{equation*}
        c_m (\pi m)^{2l} w_m + \sum \limits_{j = 1}^{m - 1}{c_j (\pi j)^{2l}
        w_j(x)} = 0.
    \end{equation*}

    Разделим на $(\pi m)^{2l}$. Тогда

    \begin{equation*}
        c_m w_m + \sum \limits_{j = 1}^{m - 1}{c_j \left(\frac{j}{m}\right)^{2l}
        w_j(x)}.
    \end{equation*}
    Заметим, что при $l \rightarrow +\infty$, правое слагаемое стремится к 0.\\
    Следовательно,

    \begin{equation*}
        c_m w_m = 0.
    \end{equation*}

    Функция $w_m(x) = \sin{(\pi m x)}$ не может принимать нулевые значения на 
    целом интервале, следовательно $c_m = 0$. Проделав те же самые рассуждения 
    для $\sum \limits_{j = 1}^{m - 1}{c_j w_j(x)}$, получаем, что

    \begin{equation*}
        c_1 = c_2 = ... = c_m = 0.
    \end{equation*}

\end{proof}

\par
\vspace{2ex}

\begin{lemma}\label{main_lemma}
    \begin{equation}
        \gamma = \inf{ \left\{ \norm{z}^2_{\omega} : z = P_mu,\enskip u \in H, 
        \enskip \norm{z} = 1 \ \right\} } > 0.
    \end{equation}
    Здесь $\norm{z}^2_{\omega} = \int\limits_{\omega}{z^2dx}$.
\end{lemma}

\begin{proof}

    Рассмотрим функцию $f$, определённую на единичной сфере в $\mathbb{R}^m$

    \begin{equation}
       f(c_1, c_2, ..., c_m) = \int \limits_{\omega} {(\sum\limits_1^m {c_jw_j})^2}
       \text{, где } c_1^2 + c_2^2 + ... + c^2_m = 1.
    \end{equation}
    По теореме Вейерштрасса, существует функция 
    $z_0 = \sum\limits_1^m {c_j^0 w_j} \in H_m$, такая, что

    \begin{equation}
       \gamma = f(c_1^0, c_2^0, ..., c_m^0) = \inf \limits_{c_1^2 + .. + c_m^2 =
       1} {f}.
    \end{equation}

    Заметим, что из линейной независимости 
    $\left\{ \omega_j \right\}^m_1$(по лемме \ref{util_lemma}) следует 
    положительность $\gamma$.

\end{proof}

\par
\vspace{2ex}

Выберем число $m \in \mathbb{N}$ так, что

\begin{equation}
    q = [(\pi(m + 1))^2 - \alpha - 1] > 0.
\end{equation}
На основании леммы \ref{main_lemma} справедлива следующая оценка
\begin{equation*}
    \frac{1}{2} \frac{d}{dt} \norm{\varphi}^2 + (\pi^2 - \alpha + r\gamma) 
    \norm{\varphi}^2 \le 0.
\end{equation*}

Далее выберем число $r = r_m > 0$ так, чтобы
$\beta = \pi^2 - \alpha + r\gamma > 0$.
Умножим обе части неравенства на $2e^{2\beta t}$, проинтегрируем по $t$ и 
получим следующую важную оценку

\begin{equation}\label{phi_mark}
    \norm{\varphi}^2 \le \norm{\varphi_0}^2 e^{-2\beta t},
\end{equation}
здесь $\varphi_0 = P_m u_0$.
\vspace{2em}

Проведём аналогичные выкладки с $\psi = Q_m u$. Домножим обе части 
\eqref{control} скалярно на $\psi$. Учтем, что

\begin{gather*}
    (u, \psi) = \norm{\psi}^2,\\*
    (u_t, \psi) = (\psi_t, \psi) = \frac{1}{2} \frac{d}{dt} \norm{\psi}^2,\\*
    (u_{xx}, \psi) = (\psi_{xx}, \psi) = -(\psi_x, \psi_x) = -
    \norm{\psi_x}^2,\\*
    (\chi_{\omega}\varphi, \psi) = \int_{\omega}{\varphi \psi} \le
    \norm{\varphi} \norm{\psi}.
\end{gather*}
С помощью неравенства Пуанкаре-Фридрихса-Стеклова получаем оценку
\begin{equation}
    \norm{\psi}^2 \le (\frac{1}{\pi(m + 1)})^2 \norm{\psi_x}^2.
\end{equation}

Тогда

\begin{equation}\label{v2}
    \frac{1}{2} \frac{d}{dt} \norm{\psi}^2 + (\pi(m + 1))^2 \norm{\psi}^2 - 
    \alpha \norm{\psi}^2 \le r \norm{\varphi} \norm{\psi}.
\end{equation}

Рассмотрим подробнее \eqref{v2}.\\
Для оценки произведения норм, воспользуемся неравенством Юнга

\begin{equation*}
    \norm{\varphi} \norm{\psi} \le (\frac{\varepsilon \norm{\psi}^2}{2} + 
    \frac{\norm{\varphi}^2}{2 \varepsilon}),
\end{equation*}
здесь $\varepsilon = \frac{2}{r}$. Из \eqref{v2} получаем неравенства

\begin{equation*}
    \frac{1}{2} \frac{d}{dt} \norm{\psi}^2 + (\pi(m + 1))^2 \norm{\psi}^2 - 
    \alpha \norm{\psi}^2 \le r (\frac{\varepsilon \norm{\psi}^2}{2} + 
    \frac{\norm{\varphi}^2}{2 \varepsilon}),
\end{equation*}
\begin{equation*}
    \frac{1}{2} \frac{d}{dt} \norm{\psi}^2  + [(\pi(m + 1))^2 - \alpha - 1] 
    \norm{\psi}^2 \le \frac{r^2}{4}\norm{\varphi}^2 \le 
    \frac{r^2}{4}\norm{\varphi_0}^2 e^{-2\beta t},
\end{equation*}
\begin{equation}
    \frac{1}{2} \frac{d}{dt} \norm{\psi}^2 + q\norm{\psi}^2 \le 
    \frac{r^2}{4}\norm{\varphi_0}^2 e^{-2\beta t}.
\end{equation}

Напомним, что $q > 0$ за счет выбора числа гармоник в операторе $P_m$.
Домножим обе части неравенства на $2e^{2qt}$ и проинтегрируем
\begin{gather*}
    \int\limits_0^t{d(e^{2q\tau}} \norm{\psi}^2) \le 
    \frac{r^2}{2} \norm{\varphi_0}^2 \frac{1}{2q - 2\beta} 
    \int\limits_0^t {e^{(2q -2\beta) \tau} d[(2q -2\beta) \tau]},\\*
    e^{2qt} \norm{\psi}^2 - \norm{\psi_0}^2 \le \frac{r^2}{2}
    \norm{\varphi_0}^2 \frac{1}{2q - 2\beta} e^{(2q -2\beta)t}.
\end{gather*}

Следовательно, 

\begin{equation}\label{psi_mark}
    \norm{\psi}^2 \le \norm{\psi_0}^2 e^{-2qt} + \frac{r^2}{2} 
    \norm{\varphi_0}^2 \frac{1}{2(q - \beta)} e^{-2\beta t}.
\end{equation}

Известно, что $u = \varphi + \psi$. Воспользуемся оценками норм $\psi$ и 
$\varphi$ [\eqref{psi_mark}, \eqref{phi_mark}] и получим

\begin{gather*}
    \norm{u}^2 = \norm{\varphi}^2 + \norm{\psi}^2 \le \norm{\varphi_0}^2 
    e^{-2\beta t} + \norm{\psi_0}^2 e^{-2qt} + \frac{r^2}{2} \norm{\varphi_0}^2 
    \frac{1}{2(q - \beta)} e^{-2\beta t}.
\end{gather*}

Окончально оценку стабилизации можно записать в виде

\begin{equation}
    \norm{u}^2 \le \norm{u_0}^2 \left( e^{-2\beta t} + e^{-2qt} + 
    \frac{r^2 e^{-2\beta t}}{4(q - \beta)} \right).
\end{equation}

    %%\subsection{Численная реализация алгоритма}
\vspace{1em}

В настоящем параграфе приведена численная реализация предложенного алгоритма 
стабилизации.\\

Рассмотрим задачу с стабилизирующим оператором

\begin{equation}\label{sys}
    u_t - u_{xx} - \alpha u = -r\chi_{\omega}P_m u, \ 0 < x < 1, \quad t > 0
\end{equation}

К уравнению \eqref{sys} добавим начальное и граничные условия

\begin{gather}\label{s_control}
    u(0, t) = u(1, t) = 0 \\*
    u(x, 0) = u_{0}(x) \in H \nonumber
\end{gather}

Для \eqref{sys} запишем разностную схему

\begin{equation}\label{scheme}
    \frac{u^{j + 1}_i - u^j_i}{\tau} - \frac{u_{i + 1}^{j + 1} - 
    2u_{i}^{j + 1} + u_{i - 1}^{j + 1}}{h^2} - \alpha u_{i}^{j + 1} + 
    r\chi_{\omega}P_m u^j_i = 0
\end{equation}

Запишем аппроксимацию начального и граничных условий

\begin{gather}
    u_i^0 = u_0(x_i) \\*
    u_1^{j+1} = u_N^{j+1} = 0 \nonumber
\end{gather}

Вспомним, что оператор проектирования имеет вид

\begin{gather*}
    P_m u = \sum \limits_{j=1}^{m} {(u, w_j) w_j} = 
    \sqrt{2} (\sum \limits_{j=1}^{m} {C_k \sin{(\pi k x)}}), \ \text{где }
    C_k = \sqrt{2} \int\limits_0^1{u(s)\sin{(\pi k s)} ds}
\end{gather*}

Заметим, что $C_k$ - это интеграл от быстро осциллирующей функции вида

\begin{equation}
    \int\limits_a^b{f(x) e^{i\omega x} dx} \approx \int\limits_a^b{L_3(x) e^{i\omega x} dx}
\end{equation}

Поскольку, функция $f$ является гладкой, то на $[a, b]$ она легко приближается 
с известной погрешностью интерполяционными многочленами. Пусть для 
определенности, это интерполяционный многочлен в форме Лагранжа

\begin{equation}
    L_3(x) = P_1(x)f(x_1) + P_2(x)f(x_2) + P_3(x)f(x_3)
\end{equation}

построенный по узлам $x_1 = a$, $x_2 = \frac{a + b}{2}$, $x_3 = b$. $P_i$ - 
многочлены второй степени, не зависящие от функции $f$. Данный метод 
приближенного интегрирования называется формулой Филона. Именно этим 
способом и будем аппроксимировать оператор $P_m$.\\

Для решения данной схемы \eqref{scheme} воспользуемся методом прогонки.

    %%\subsection{Примеры численного моделирования стабилизации неустойчивых
решений уравнения теплопроводности}

\vspace{1em}

\newtheorem{exmp_st}{Пример}

\begin{exmp_st}
\end{exmp_st}

В качестве начальных условий возмем $u_0 = sin(\pi x)$. Продемонстрируем 
стабилизацию системы при $\alpha = \pi^2 + 0.1$. Фиксируем $\omega = [0, 0.2]$. 
Необходимо подобрать параметры $m$, $r_m$ так, чтобы $\beta > 0$ и $q > 0$. 
Рассмотрим подробнее $q = [(\pi(m + 1))^2 - \alpha - 1]$. При заданном 
$\alpha = \pi^2 + 0.1$, достаточно взять $m = 2$ для выполнения неравенства. 
Параметр $r$ придется подобрать так, чтобы решение стремилось к нулю.


\begin{figure}[H]
    \centering
    \includegraphics[width=4in]{par_re_pi01}
    \caption{Управление $m = 2,\; r = 8$}
    \label{fig:test2}
\end{figure}

\begin{exmp_st}
\end{exmp_st}

Пусть $u(x, 0) = x(1 - x)$ - начальное условие . Заведомо выберем параметр 
$\alpha = \pi^2 + 3$ большим. Зафиксируем $\omega = (0, 0.4)$. 
Необходимо подобрать $m$, таким чтобы $q > 0$. При $m \ge 2$ условие выполняется, 
поэтому мы фиксируем $m = 2$. На рис.4 показано, как быстро растет решение 
задачи \eqref{sys} - \eqref{s_control} при небольшом увеличении $\alpha$. 
Стабилизация этой системы представлена на рисунке 5

\begin{figure}[H]
    \centering
    \includegraphics[width=4in]{par_re_pi3}
    \caption{Управление $m = 2,\; r = 15$}
    \label{fig:test2}
\end{figure}

%%    \section{Стабилизация неустойчивых стационарных решений уравнения Бюргерса}
\vspace{1em}

\subsection{Постановка задачи}

Рассмотрим уравнение Бюргерса с вязкостью $\nu > 0$ на интервале $\Omega = (0,
1) \subset \mathbb{R}$

\begin{equation}\label{burger}
    u_t - \nu u_{xx} + u_x u = f + y, \ u|_{\Gamma} = u_b, \quad t > 0
\end{equation}

Функция $f = f(x)$, $x \in \Omega$ является заданной, а функция $y = y(x, u)$
рассматривается как управление, носитель которого при фиксированном $t > 0$
содержиться в $\bar{\omega}$, где $\bar{\omega}$ - заданная подобласть $\Omega$.
Через $\Gamma = \{0, 1\}$ обозначена граница $\Omega$, $u_b \in \mathbb{R}^2$\\

Пусть $U$ - стационарное решение \eqref{burger}, то есть

\begin{equation}\label{stationary_sol}
    -\nu U_{xx} + U U_x = f, \ U|_{\Gamma} = u_b
\end{equation}

и $U$ является неустойчивой особой точкой динамической системы, порождаемой
эволюционным уравнение \eqref{burger} в фазовом пространстве $H = L^2(\Omega)$.
Задача стабилизации состоит в следующем:\\

\textit{Для заданного} $\sigma > 0$ 
\textit{требуется найти оператор управления с обратнойсвязью} 
$y = \Lambda(u - U) : H \to H$ \textit{такой, что} $\mathbf{supp}y(\cdot,t) \subset 
\bar{\omega}$ \textit{и решение замкнутой системы}

\begin{equation}
    u_t - \nu u_xx + u u_xx = f + \Lambda(u - U), \ u|_{\Gamma} = u_b,
    \ t > 0, \quad u|_{t=0} = u_0
\end{equation}

\textit{сходится к} $U$ \textit{с заданной скоростью} $\sigma$

\begin{equation}
    \norm{u(t) - U}_{L^2(\Omega)} \le C e^{-\sigma t} \ \text{при } t
    \to +\infty
\end{equation}

\textit{если величина} $\norm{u_0 - U}_{L^2(\Omega)}$ \textit{достаточно мала}\\

Пусть $\varphi = u - U$, тогда

\begin{gather}\label{fluct}
    \varphi_t - \nu \varphi_{xx} + \varphi U_x + (\varphi + U)\varphi_x =
    \Lambda(\varphi)\\* 
    \varphi|_{\Gamma} = 0, \ t > 0\\*
    \varphi|_{t = 0} = \varphi_0 = u_0 - U
\end{gather}

Требуется чтобы $\norm{u(t) - U}_{L^2(\Omega)} \le C e^{-\sigma t}$ при $t \to
+\infty$, если мала норма $\norm{\varphi_0}_{L^2(\Omega)}$

\subsection{Неустойчивость стационарных решений shock-like}

Рассмотрим семейство стационарных решений shock-like

\begin{equation}\label{shock_like}
    U(x) = -2\sigma\tanh{(\sigma(x - \frac{1}{2}))}, \ \text{где } \sigma \ge 0
\end{equation}

\begin{figure}[H]
    \centering
    \includegraphics[width=4in]{fig1}
    \caption{$U(x)$ при разных $\sigma$}
\end{figure}

Для изучения устойчивости системы \eqref{fluct}, мы линеаризуем её

\begin{gather}\label{linearized}
    \theta_t = \theta_{xx} + 2 \sigma (\tanh(\sigma(x - \frac{1}{2}))\theta)_x \\*
    \theta(0, t) = \theta(1, t) = 0,
\end{gather}

где $\theta(x, t)$ - решение уравнения \eqref{linearized}, которое является
линеаризацией \eqref{fluct}. Заметим что \eqref{linearized} является  уравнение 
конвенкции-диффузии-реакции. Для простоты изучения устойчивости, мы избавимся 
от конвекционого члена используя преобразование 
$\zeta(x, t) = G(x)\theta(x, t)$, где 

\begin{equation}
    G(x) = \frac{\cosh(\sigma(x - \frac{1}{2}))}{\cosh(\frac{\sigma}{2})}
\end{equation} 

Имеем 

\begin{gather} \label{transf_linear}
    \zeta_t = \zeta_{xx} + \sigma^2 \left( \frac{2}{\cosh^2(\sigma(x - \frac{1}{2}))} - 1 \right) \zeta \\* 
    \zeta(0) = \zeta(1) = 0 
\end{gather}

Для $\sigma = 0$ система нейтрально устойчива. Для $\sigma > 0$, член 
$\left(\frac{2}{\cosh^2(\sigma(x - \frac{1}{2}))} - 1 \right)$  в 
\eqref{transf_linear} также является неустойчивым в окрестности 
$x = \frac{1}{2}$ (Рис 2.), т.е. положительность этого члена ведет к
неустойчивости системы


\begin{figure}[H]
    \centering
    \includegraphics[width=4in]{fig2}
    \caption{Значение реакционного члена в \eqref{transf_linear}}
\end{figure}

    %%\subsection{Примеры неустойчивых стационарных решений}

Приведем численные примеры, демонстрирующие зависимость устойчивости
стационарного решения \eqref{shock_like} от параметра $\sigma > 0$ и от
"возмущающих" гармоник $\theta_0$

\newtheorem{exmp_bur}{Пример}
\begin{exmp_bur}
\end{exmp_bur}

Начальное условие $\theta_0(x) = \frac{\sin(\pi x)}{G(x)}$, параметр $\sigma$ 
возмем равный 15. На рисунке 3 показано неустойчивое поведение системы
\eqref{fluct}

\begin{figure}[H]
    \centering
    \includegraphics[width=4in]{ex_s15}
    \caption{Без управления}
\end{figure}

\begin{exmp_bur}
\end{exmp_bur}
Теперь начальное условия $\theta_0(x) = \frac{x^2}{G(x)}$. Параметр $\sigma = 15$. 

\begin{figure}[H]
    \centering
    \includegraphics[width=4in]{ex_x2_s15}
    \caption{Без управления}
\end{figure}

    %%\section{Введение}

Пусть

\begin{equation}
    \Omega = (0, 1), \quad \omega \subset \Omega
\end{equation}

Неустойчивое уравнение Бюргерса

\begin{gather}
    y_t - \nu y_{xx} + yy_x = f \\*
    y|_{\partial \Omega} = y_{bi} \\*
    y|_{t=0} = y_0
\end{gather}

Стационарное уравнение Бюргерса

\begin{gather}
    y_{sxx} + y_s y_{sx} = f_s \\*
    y|_{s \partial \Omega} = y_{bi}
\end{gather}

\subsection{Задача стабилизации}
Найти оператор управления с обратной связью $\Lambda$ такой, что решение задачи
(1), где $f = f_s + \Lambda (y - y_s)$ экспоненциально сходится к $y_s$

Обозначим за $\varphi = y - y_s$, тогда

\begin{gather}
    \varphi_t - \nu \varphi_{xx} + y_s \varphi_x + \varphi y_{sx} + \varphi
    \varphi_x = \Lambda(\varphi) \\*
    \varphi_0 = y_0 - y_s \\*
    \varphi|_{\partial \Omega} = 0
\end{gather}

\subsection{Формализация}
Здесь и далее
\begin{equation}
    H = L^2(\Omega), \quad V = H^1_0(\Omega), \quad \norm{z}^2 = \int_{\Omega}{z^2dx}
\end{equation}

\begin{equation}
    V \subset H = H^{'} \subset V^{'} = H^{-1}(\Omega), (y, z) =
    \int_{\Omega}{yzdx}
\end{equation}

\begin{equation}
    A: V \rightarrow V^{'}, (Ay, z) = (y_x, z_x)
\end{equation}

\begin{equation}
    B: V \times V \rightarrow V^{'}, \quad (B(y, z), w) = (yz_x, w)
\end{equation}

\begin{equation}
    A w_j = \lambda_j w_j, \quad j = 1, 2.. \quad \lambda_j \rightarrow +\infty
\end{equation}

\begin{equation}
    \lambda_j = (\pi j)^2, \quad w_j(x) = \sqrt{2}\sin{(\pi j x)}
\end{equation}

Заметим что базис ортонормирован в $H$

\begin{equation}
    (w_j, w_k) = \delta_{jk}
\end{equation}

Введем следующие обозначения

\begin{equation}
    H_m = span{w_1, ..., w_m}, \quad P = P_m: H \rightarrow H_m, \quad Q=Q_m = I
    - P
\end{equation}

\subsubsection{Неравенства}

\begin{equation}
    \lambda_1 \norm{v}^2 \le \norm{v_x}^2, \quad \norm{v}^4_{L^4} \le \norm{v}^2
    \norm{v_x}^2
\end{equation}

\begin{equation}
    |(B(y, z), w)| \le \norm{y}_{L^4} \norm{z_x} \norm{w}_{L^4} \le
    \norm{y}^{1/2} \norm{y_x}^{1/2} \norm{z_x} \norm{w}^{1/2} \norm{w_x}^{1/2}
\end{equation}

\begin{equation}
    \norm{w}^2 \le \lambda^{-1}_{m + 1} \norm{w_x}^2 \forall w \in V \cap
    H_m^{\bot}
\end{equation}

\begin{equation}
    |v(x)| \le \norm{v_x}, \quad |(B(y, z), w)| \le \norm{y_x} \norm{z_x}
    \norm{w} \text{или} \norm{y} \norm{z_x} \norm{w_x}
\end{equation}

Оператор локально распределенного управления $\Lambda: H \rightarrow H$, \\
\begin{equation}
    \Lambda(\varphi) = -r \xi_{\omega}P\varphi
\end{equation}


Здесь \\
\begin{gather*}
    \begin{matrix}
        r & =
        & \left\{
        \begin{matrix}
            r_0, & 2 \norm{Q\varphi} \le \norm{P\varphi}, \quad r_0 > 0, \\
            0, & \mbox{иначе. }
        \end{matrix} \right.
    \end{matrix}, \quad
    \begin{matrix}
        \chi_{\omega}(x) & =
        & \left\{
        \begin{matrix}
            0, & \mbox{если } x \notin \omega, \\
            1, & \mbox{иначе. }
        \end{matrix} \right.
    \end{matrix}
\end{gather*}

Перепишем задачу в следующем виде

\begin{gather}
    \varphi' + \nu A\varphi + B(\varphi, y_s) + B(y_s + \varphi, \varphi) =
    \Gamma(\varphi), \quad t > 0 \\*
    \varphi(0) = \varphi_0 = y_0 - y_s
\end{gather}

\subsection{Разрешимость задачи на конечном интервале времени}

\newtheorem{theorem}{Теорема}

\begin{theorem}
    Пусть $y_s \in H^1, \quad \varphi_0 \in H, \quad T > 0$. Тогда $\exists$ решение
    задачи $\varphi \in C([0, T], H) \cap L^2(0, T, V), \varphi' \in L^2(0, T, V')$
\end{theorem}

\begin{remark}
    Пусть $h(\lambda)$ единичная функция Хевисайда, \\
    \begin{gather*}
        \begin{matrix}
            h(\lambda) & =
            & \left\{
            \begin{matrix}
                1, & \lambda \ge 0 \\
                0, & \lambda < 0
            \end{matrix} \right.
        \end{matrix}
    \end{gather*}

    , тогда оператор управления примит следующий вид

    \begin{equation}
        \Lambda(\varphi) = -r_0 h(-2\norm{Q\varphi} +
        \norm{P\varphi})\chi_{\omega}P\varphi
    \end{equation}

\end{remark}


\begin{proof}
    Пусть $\Omega > 0$. Выберим параметры $\sigma$, $r_0$, $m$ такие, что
    $\norm{g(t)}$ ограничена на $[0, +\infty)$. Здесь $g(t) = \varphi(t)e^{\sigma t}$
    \begin{gather*}
        g' + \nu Ag - \sigma g + B(g, y_s) + B(y_s + ge^{-\sigma t}, g) = -r_0 +
        h(\norm{Pg} - 2 \norm{Qg}) + \chi_{\omega}Pg\\*
        g(0) = \varphi_0
    \end{gather*}
    Пусть $p = Pg$, $q = Qg$. Рассмотрим интервал $(t_0, t_1)$, где $\norm{p(t)}
    \le 2 \norm{q(t)}$:
    \begin{gather*}
        g' + \nu Ag - \sigma g + B(g, y_s) + B(y_s + ge^{-\sigma t}, g) = 0, \quad
        t_0 < t < t_1\\
        \frac{1}{2}\frac{d}{dt}\norm{g}^2 + \norm{g_x}^2 - \sigma \norm{g}^2 +
        \frac{1}{2}(y_{sx}, g^2) = 0
    \end{gather*}

    \begin{gather*}
        \norm{g}^2 = \norm{p}^2 + \norm{q}^2 < 5\norm{q}^2 < 5\lambda^{-1}_{m +
        1}\norm{q_x} \le 5 \lambda^{-1}_{m + 1}\norm{g_x}^2
    \end{gather*}

    Пусть 
    \begin{equation}
        C_1 = \sigma + \frac{1}{2}max|y_{sx}|, \quad 5C_1\lambda^{-1}_{m + 1} <
        \frac{\nu}{2}
    \end{equation}

    Тогда отсюда 
    \begin{equation}
        \frac{d}{dt}\norm{g}^2 + \nu \norm{g_x} < 0
    \end{equation}

    Тогда следует

    \begin{equation}
        \norm{g(t_1)} < \norm{g(t_0)}
    \end{equation}

    Далее 
    \begin{gather*}
        \frac{1}{2}\frac{d}{dt}\norm{q}^2 + \nu \norm{q_x}^2 - \sigma \norm{q}^2 +
        ((y_s g)_x, q) + \frac{1}{2}e^{-\sigma t}((g^2)_x, q) = 0\\
        \norm{g_x}^2 = \norm{p_x}^2 + \norm{q_x}^2 \le \lambda_{m} \norm{p}^2 +
        \norm{q_x}^2 \le 5\norm{q_x}^2\\
        |(y_s g)_x, q)| \le C_0 \norm{g} \norm{q_x} < 5 C_0 \lambda^{-1}_{m + 1}
        \norm{q_x}^2, \quad C_0 = max|y_s|\\
        |((g^2)_x, q)| = |(g^2, q_x)| \le \norm{g} \norm{g_x} \norm{q_x}
    \end{gather*}
    
    Пусть теперь $\norm{p(t)} \ge 2 \norm{q(t)}, \quad t \in (t_0, t_1)$\\
    \begin{gather*}
        g' + \nu Ag - \sigma g + (y_s g)_x + \frac{1}{2} e^{-\sigma t} (g^2)_x +
        r_0 \chi_{\omega} P = 0
    \end{gather*}

    Заметим что
    \begin{gather*}
        \gamma |(p, q)_{L^2(\omega)}| \le \norm{p} \norm{q} \le \frac{1}{2}
        \norm{p}^2 \text{следует, что} \\
        \norm{p}^2_{L^2(\omega)} + \gamma (p,
        q)_{L^2(\omega)} \ge \frac{\gamma}{2}\norm{p}^2
    \end{gather*}
    
    \begin{gather*}
        \frac{1}{2} \frac{d}{dt} [ \norm{p}^2 + \nu [ \norm{p_x}^2 - \sigma [
        \norm{p}^2 + \gamma \norm{q}^2] ]  + ( (y_s g)_x, p + \gamma q) + \gamma
    \norm{q} \\* + \gamma\norm{q_x}^2] + \frac{1}{2} e^{-\sigma t} ( (g^2)_x, p +
    \sigma q) + \frac{r_0 \gamma}{2} \norm{p}^2 \le 0\\
    |( (y_s g)_x, p)| \le C_0 \norm{p_x} \norm{g} \le \frac{3}{2} C_0 \norm{p}
    \norm{p_x} \le \frac{\nu}{2} \norm{p_x}^2 + \frac{1}{2\nu}(\frac{3}{2}C_0)^2
    \norm{p}^2
    \end{gather*}

    \begin{equation}
        \norm{g} \le \norm{p} + \norm{q} \le \frac{3}{2} \norm{p}
    \end{equation}

    \begin{gather*}
        |((y_s g)_x, q)| \le C_0 \norm{q_x} \norm{g} \le \frac{3}{2} C_0
        \norm{q_x} \norm{p} \le \frac{\nu}{2} \norm{q_x}^2 + \frac{9C_0^2}{8\nu}
        \norm{p}^2
    \end{gather*}

    \begin{align*}
        \frac{d}{dt}(\norm{p}^2 + \gamma \norm{g}^2) + \nu(\norm{p_x}^2 + \gamma
        \norm{q_x}^2) + (r_0\gamma - (2 + \gamma)\sigma - (1 +
        \gamma)\frac{9C_0^2}{4\nu})\norm{p}^2 \\* \le |(g^2, p_x + \gamma q_x)|
    \end{align*}

    \begin{equation}
        r_0 > C_2 = (1 + \frac{2}{\gamma})\sigma + (1 +
        \frac{1}{\gamma})\frac{9C_0^2}{4\nu}
    \end{equation}

    Оценим правую часть 

    \begin{gather*}
        |(g^2, p_x + \gamma q_x)| = (1 - \gamma)|(g^2, p_x)| = (1 - \gamma)|(p^2
        + 2pq + q^2, p_x)| \\*
        = (1 - \gamma)|(q^2, p_x) - (p^2, q_x)| \le \norm{p_x} \norm{q_x}
        (\norm{p} + \norm{q}) \\* \le
        \frac{1}{2\sqrt{\gamma}}(\norm{p_x}^2 + \gamma \norm{q_x}^2)
        \frac{2}{\sqrt{\gamma}}\sqrt{\norm{p}^2 + \gamma \norm{q}^2}
    \end{gather*}
    т.к. $\norm{p} + \norm{q} \le \frac{2}{\sqrt{\gamma}}\sqrt{\norm{p}^2 +
        \gamma \norm{q}^2}$ отсюда следует, что

    \begin{equation}
        \frac{d}{dt}(\norm{p}^2 + \gamma \gamma{q}^2) + (\norm{p_x}^2 + \gamma
        \norm{q_x}^2) (\nu - \frac{1}{\nu} \sqrt{\norm{p}^2 + \gamma \norm{q}^2}
        \le 0
    \end{equation}

    
\end{proof}

    %%%\input{num_implementation_bur}
    %%\subsection{Примеры численного моделирования стабилизации неустойчивых решений}
\vspace{1em}

\subsubsection{Стабилизация неустойчивых решений}

\newtheorem{exmp_stbur}{Пример}

\begin{exmp_stbur}
\end{exmp_stbur}

Пусть $\theta_0 = \frac{sin(\pi x)}{G(x)}$ - начальное условие. Рассмотрим
стационарное решение уравнения Бюргерса с параметром $\tau$ возмем равный 15. 
В предыдущем параграфе показано неустойчивое поведение системы без управления. 
Cтабилизация указанного решения за счет управления с параметрами 
$\omega = (0, 0.4), \; r = 30, \; m = 4$ предоставлена ниже


\begin{figure}[H]
  \centering
  \includegraphics[width=3.5in]{re_s15}
  \caption{Управление с $\omega = (0, 0.4), \; m = 4, \; r = 30$}
  \label{fig:test2}
\end{figure}


\begin{exmp_stbur}
\end{exmp_stbur}
Пусть теперь начальное условие системы $\theta_0(x) = \frac{x^2}{G(x)}$.
Впспомним, что чем больше параметр $\tau > 0$, тем сильнее он влияет на
неустойчивость нашей системы \eqref{fluct}. Зафиксируем следующие параметры
стабилизирующего оператора управления $\omega = (0, 0.4), \ r = 30$, а параметр 
$\tau$  возмем равный $15$. Неустойчивость этой системы предемонстирована в
предыдущем параграфе (рис 2). Ниже представлен процесс стабилизации

\begin{figure}[H]
 \centering
  \includegraphics[width=3.5in]{re_x2_s15}
  \caption{$\omega = (0, 0.4), \; r = 30, \; m = 3$}
  \label{fig:test2}
\end{figure}

\subsubsection{Анализ параметров оператора стабилизации}

Рассмотрим ....


    %%%\section*{Заключение}
%\addcontentsline{toc}{section}{Заключение}
\Conclusion
\vspace{2em}

В представленной ВКР предложен простой метод стабилизации неустойчивых
стационарных решений параболических уравнений. Особенность предложенного
алгоритма заключается в локальности носителя управления и конечномерности образа
оператора управления. В работе получены следующие результаты:

\begin{itemize}

\item{Разработан алгоритм реализации стабилизирующего
управления для линейного уравнения теплопроводности и нелинейного уравнения
Бюргерса;}

\item{Проведён теоретический анализ устойчивости неустойчивых параболических
систем;}

\item{Реализован алгоритм стабилизации;}

\item{Приведены примеры неустойчивых стационарных решений, примеры численного 
моделирования стабилизации неустойчивых систем.}

\end{itemize}
%разработан алгоритм реализации стабилизирующего
%управления для линейного уравнения теплопроводности и нелинейного уравнения
%Бюргерса, теоретическое обоснование методов стабилизации, в том числе написание программы для решения разностных схем и
%реализация метода Филона для интегралов от быстро осциллирующих функций.
%Приведены примеры численного моделирования неустойчивых стационарных решений
%линейных параболических уравнений на примере уравнения теплопроводности и нелинейных для неустойчивых
%стационарных решений уравнения Бюргерса типа "shock-like", и применение
%предложенного алгоритма стабилизации.

    %%\section{Список литературы}
\vspace{2em}

\begingroup
\renewcommand{\section}[2]{}%
%\renewcommand{\chapter}[2]{}% for other classes
\begin{thebibliography}{}
\bibitem{}
	А.Ю. Чеботарёв, Конечномерная стабилизация с заданой скоростью систем типа Навье-Стокса. Дальневосточный мат. журнал 2010. Том 10 $№2$ с 200-205

\bibitem{}
  Miroslav Krstic, Lionel Magnis, Rafael Vazquez, Nonlinear Control of the Burgers PDE—Part I: Full-State Stabilization.\ June 11-13, 2008

\bibitem{}
  Alexei Kourbatov, Lyapunov Exponents for Burgers’ Equation. Sbornik Nauchnykh Trudov MFTI. Moscow, 1992

\bibitem{oe04}
    M. Krstic, A. Smyshlyaev, Boundary Control of PDEs.\ 2008
\bibitem{}
	M. Krstic, A. Smyshlyaev, Boundary Control of PDEs (short course).\ 2006
\bibitem{}
    А. В. Фурсиков. Стабилизируемость квазилинейного параболического уравнения с помощью граничного управления с обратной связью. Матем. сб., 192:4 (.\ 2001)
\bibitem{}
	O.А. Ладыженская, Краевые задачи математической физики.\ 1973
\bibitem{}
	А.А. Самарский, А.В. Гулин, Численные методы. общий базовый курс.\ 1989
\bibitem{}
	Метод прогонки, неявная схема [Электронный ресурс] : \url{http://ikt.muctr.ru/html2/4/lek4_2_1.html}
\bibitem{}
	Формула Филона [Электронный ресурс] : \url{http://window.edu.ru/resource/886/19886/files/rsu177.pdf}

\end{thebibliography}
\endgroup
%\end{onehalfspace}

\frontmatter % Выключаем нумерацию всего

\tableofcontents

%\chapter*{Введение}
%\addcontentsline{toc}{section}{Введение}
%\vspace{1em}

\Introduction

В реальных физических процессах неизбежно возникают непредусмотренные 
флуктуации, и поэтому возникает необходимость разработки методов построения 
управлений, способных реагировать на непредусмотренные возмущения и подавлять 
их \cite{Furs}. Проблемы стабилизации систем параболического типа привлекают внимание 
специалистов в силу прикладной значимости данной системы \cite{Chebotarev}. 
Управления такого типа называются \emph{управлениями с обратной связью} \cite{KS}.

Вопрос о стабилизируемости различных эволюционных уравнений в частных 
производных с помощью управлений исследовался многими авторами, среди которых 
A.Kwiecinka \cite{KWCK}, Barbu V. \cite{Barbu}, M. Kristic
\cite{KMV, KS}, Фурсиков А.В. \cite{Furs}, Чеботарёв А.Ю. 
\cite{Chebotarev, ChebotarevBS, ChebotarevMGT} и разработаны методы управления, 
такие как : стабилизация Ляпунова, метод backstepping, управление с обратной
связью. Существующие методы стабилизации довольно громоздкие и требуют значительных 
вычислительных затрат, тогда как алгоритм рассматриваемый в ВКР прост в реализации.

Представленная ВКР посвящена изучению алгоритмам стабилизации для неустойчивых
по Ляпунову линейных и нелинейных параболических систем. В работе
рассматривается алгоритм стабилизация эволюционных систем с постоянными коэффициентами 
конечномерным локальным управлением с обратной связью. Указанный
метод позволяет провести стабилизацию с любой наперёд заданной скоростью $\sigma
> 0$ засчет выбора параметров стабализирующего оператора.

В работе проделаны теоретический анализ устойчивости линейной параболической
системы на примере уравнения теплопроводности, численные примеры иллюстрирующие
неустойчивые решения, примеры численного моделирования стабилизации, теоретическое
обоснование алгоритма стабилизации, предложен метод стабилизации неустойчивых
стационарных решений типа "shock-like" для вязкого уравнения Бюргерса.

%\section*{Основные обозначения}
%\addcontentsline{toc}{section}{Основные обозначения}
%\vspace{3em}

\Abbreviations

\begin{description}

    \item{Символы $u_t, u_x, u_{xx}$ обозначают соотвествующие классические
        производные функции $u$}

    \item{Через $E'$ обозначим пространство сопряженное к пространству $E$}

    \item{$L^{p}(\Omega), \ p \ge 1,$ - банахово пространство (т.е. полное линейное
нормированное пространство), состоящее из всех определенных и измеримых (по
Лебегу) на $\Omega$ функций, имеющих конечную норму
\begin{equation*}
    \norm{u}_{p, \Omega} = (\int\limits_{\Omega}{|u|^p dx})^{1/p}
\end{equation*}
Норму в $L^2(\Omega)$ обозначим коротко $\norm{\ .\ }$, а скалярное произведение
как $(\cdot,\cdot)$. Этим же символом будем обозначать отношения двойственности
между $E$ и $E'$.}\cite{OAL}

\item{$W^l_m(\Omega)$(Пространство Соболева) - функциониональное пространство,
состоящее из функций из пространства Лебега, имеющих обобщенные производные 
заданного порядка из $L^m(\Omega)$. В частности
$W^1_2(\Omega)$ состоит из элементов $L^2(\Omega)$, имеющих квадратично
суммируемые по $\Omega$ обобщенные производные первого порядка. В дальнейшим,
через $H^l(\Omega)$ обозначим $W^l_2(\Omega)$.}

\item{Здесь и далее, $\Omega = (0, 1) \subset \mathbb{R}$, $H = L^2(\Omega)$ и 
    $V = H^1_0(\Omega)$}

\end{description}




\mainmatter

\section{Стабилизация неустойчивых параболических систем}

\subsection{Анализ устойчивости линейного параболического уравнения}

Рассмотрим параболическое уравнение

\begin{equation}\label{dif_form}
    u_t = u_{xx} + \alpha u, \ 0 < x < 1, \ t > 0
\end{equation}

с начальным и граничными условиями:

\begin{gather}\label{d_control}
    u(0, t) = u(1, t) = 0, \\*
    u(x, 0) = u_{0}(x) \in L^2(0, 1). \nonumber
\end{gather}

Здесь и далее через $u_t$, $u_x$, $u_{xx}$ .. обозначаются соотвествующие
частные производные функции $u$.\\
Умножим уравнение \eqref{dif_form} на $u$ скалярно в $L^2(0, 1)$

\begin{equation*}
    (u_t, u) = (u_{xx}, u) + \alpha (u, u).
\end{equation*}

Скалярное произведение в $L^2(0, 1)$ определяется как $(u, v) = \int_0^1 uv dx$,
а норма как $\norm{u} = \sqrt{(u, u)}$. Получаем

\begin{equation}\label{int_form}
    \frac{1}{2}\frac{d}{dt}\norm{u}^2 = -\norm{u_x}^2 + \alpha \norm{u}^2.
\end{equation}

C помощью неравенства Пуанкаре–Фридрихса-Стеклова

\begin{equation*}
    \norm{u}^2 \le \frac{1}{\pi^2} \norm{u_x}^2.
\end{equation*}
получаем следующую оценку 
\begin{equation}\label{stable_opr}
    \frac{1}{2}\frac{d}{dt}\norm{u}^2 \le (\alpha - \pi^2)\norm{u}^2.
\end{equation}
Рассмотрим 3 случая

\begin{enumerate}
    \item $\alpha = \pi^2$. Тогда из \eqref{stable_opr} следует неравенство
        \begin{equation}
            \norm{u}^2 \le \norm{u_0}^2.
        \end{equation}
        
        Указанное неравенство означает, что нулевое решение задачи
        \eqref{dif_form} - \eqref{d_control} устойчиво по Ляпунову, но не устойчиво ассимптотически
    
    \item $\alpha < \pi^2$. Обозначим $\frac{\mu}{2} = -(\alpha - \pi^2)$.\\
        
        Тогда

        \begin{equation}\label{less_pi2}
            \frac{d}{dt}\norm{u}^2 + \mu \norm{u}^2 \le 0
        \end{equation}

        Домножим обе части \eqref{less_pi2} на $e^{\mu t}$. Тогда, 
        $\frac{d(\norm{u}^2 e^{\mu t})}{dt} \le 0$. Проинтегрируем и в итоге получим

        \begin{equation*}
            \norm{u}^2 \le \norm{u_0}^2 e^{-\mu t} = \norm{u_0}^2 e^{2(\alpha - \pi^2) t}
        \end{equation*}

        Полученная оценка гарантирует ассимптитическую экспоненциальную устойчивость.

    \item $\alpha > \pi^2$. Решение начально краевой задачи \eqref{dif_form} - \eqref{d_control} имеет вид
        \begin{equation}
            u(x, t) = 2 \isum{a_j e^{(\alpha - \pi^2 j^2)t}\sin{(\pi j x)}}
        \end{equation}

        Здесь $a_j = \int_0^1{u_0 \sin{(\pi j s)} ds}$. Первый член суммы указывает на темп роста решения при $t \rightarrow \infty$. Следовательно, система неустойчива.
\end{enumerate}

Для стабилизации системы \eqref{int_form} в случае 3, будем использовать ниже описанный метод.

\subsection{Численные примеры неустойчивых решений уравнения теплопроводности}
\vspace{1em}

\newtheorem{exmp}{Пример}

\begin{exmp}
\end{exmp}

В качестве начальных условий выберем $u_0 = \sin(\pi x)$. Рис. 1. иллюстрирует 
неустойчивость нулевого решения при $\alpha = \pi^2 + 0.1$.

\begin{figure}[H]
    \centering
        \includegraphics[width=3.5in]{par_ex_pi01}
        \caption{}
        \label{fig:test1}
\end{figure}

\begin{exmp}
\end{exmp}

На рис. 2 приведен график решения задачи \eqref{dif_form} - \eqref{d_control}
при $\alpha = \pi^2 + 3$ и $u_0(x) = x(1 - x)$, который демонстрирует 
экспоненциальный рост решения при увеличении $t$.

\begin{figure}[H]
    \centering
        \includegraphics[width=3.5in]{par_ex_pi3}
        \caption{}
        \label{fig:test1}
\end{figure}


%\section{Стабилизация конечномерным локальным управлением с обратной связью}
%\vspace{1em}
\section{Стабилизация конечномерным локальным управлением с обратной связью}

Рассмотрим систему

\begin{equation}\label{ndif_form}
    u_t = u_{xx} + \alpha u, \ x \in \Omega, \ t > 0.
\end{equation}
с начальным и граничными условиями
\begin{gather}
    u(0, t) = u(1, t) = 0, \\*
    u(x, 0) = u_{0}(x) \in H .\nonumber
\end{gather}

Как показано в \S 1, в случае, когда $\alpha > \pi^2$, нулевое решение уравнения 
\eqref{ndif_form} неустойчиво.
Задача стабилизации параболического уравнения, заданного в ограниченной области,
заключается в построении такого оператора управления, чтобы решение смешанной 
краевой задачи стремилось (при $t \rightarrow \infty$) к заданному стационарному 
решению с предписанной скоростью $e^{(-\delta_0t)}$.

Сформулируем задачу стабилизации неустойчивого нулевого решения уравнения 
\eqref{ndif_form} за счет локального управления.\\

Пусть $\omega \subset \Omega$ - произвольный интервал такой, что 
$\bar{\omega} \subset \Omega$.Задача стабилизации за счет конечномерных локально 
распределённых в $\omega$ управлений заключается в построении оператора 
$\mathcal{R} : H \rightarrow H$ такого, что

\begin{enumerate}
    \item $\forall z \in H \ \mathbf{supp} \ \operator{z} \subset \omega$,
    \item $\dim \operator{(H)} < +\infty$.
\end{enumerate}
и при этом решение задачи
\begin{gather}
    u_t - u_{xx} - \alpha u = \operator{u}. \nonumber\\
    u(0, t) = u(1, t) = 0, \ u(x, 0) = u_0. \nonumber
\end{gather}
экспоненциально стремится к нулю при $t \rightarrow + \infty$.

\section{Конструкция оператора управления}
\vspace{1em}

Заметим, что функции

\begin{equation}\label{basis}
    w_j = w_j(x) = \sqrt{2}\sin{(\pi j x)}, \ x \in \Omega, \ j=1, 2, ..
\end{equation}
образуют базис в $H$ и в $V$, причем в $H$ базиc ортонормирован.\\
Через $H_m$ обозначим подпространство в $H$, образованное первыми $m$ функциями 
из \eqref{basis}.\\

Далее рассмотрим следующие операторы проектирования

$$P_m : H \rightarrow H_m, \ Q_m : H \rightarrow H_m^{\perp}.$$

\begin{equation}
    P_m u = \sum \limits_{j=1}^{m} {(u, w_j) w_j}.
\end{equation}

\begin{equation}
    Q_m u = (I - P_m)u(x) = \sum \limits_{j=m + 1}^{\infty} {(u, w_j) w_j}.
\end{equation}

В качестве оператора стабилизиции будем рассматривать следующий конечномерный
оператор

$$\operator{z} = -r\chi_{\omega}P_mz, \ r > 0,$$
здесь
\begin{gather*}
    \begin{matrix}
        \chi_{\omega}(x) & =
        & \left\{
        \begin{matrix}
            0, & \mbox{если } x \notin \omega, \\
            1, & \mbox{иначе. }
        \end{matrix} \right.
    \end{matrix}
\end{gather*}

В следующем параграфе будет доказано, что существуют подходящие параметры 
$m \in \mathbb{N}$, $r = r_m > 0$, при которых $\operator{}$ обеспечивает 
стабилизацию неустойчивого решения.

\section{Теоретическое обоснование стабилизации}
\vspace{1em}

Рассмотрим уравнение
\begin{gather}\label{control}
    u_t - u_{xx} - \alpha u = -r\chi_{\omega}\varphi,\\*
    u|_{x = 0;1} = 0.
\end{gather}
Здесь $\varphi = P_mu$. Домножим скалярно обе части \eqref{control} на
$\varphi$. Учтем, что

\begin{gather*}
    (u, \varphi) = \norm{\varphi}^2,\\*
    (u_t, \varphi) = (\varphi_t, \varphi) = \frac{1}{2} \frac{d}{dt}
    \norm{\varphi}^2, \\*
    (u_{xx}, \varphi) = (\varphi_{xx}, \varphi) = -(\varphi_x, \varphi_x) = -
    \norm{\varphi_x}^2, \\*
    (\chi_{\omega}\varphi, \varphi) = \norm{\varphi}^2_{\omega}.
\end{gather*}
Тогда
\begin{equation*}
    \frac{1}{2} \frac{d}{dt} \norm{\varphi}^2  + \norm{\varphi_x}^2 - 
    \alpha \norm{\varphi}^2 + r \norm{\varphi}^2_{\omega} = 0.
\end{equation*}
Воспользуемся неравенством Пуанкаре-Фридрихса-Стеклова

\begin{equation}
    \norm{\varphi}^2 \le \frac{1}{\pi^2} \norm{\varphi_x}^2.
\end{equation}
В итоге получаем

\begin{equation}
    \frac{1}{2} \frac{d}{dt} \norm{\varphi}^2 + \pi^2 \norm{\varphi}^2 - 
    \alpha \norm{\varphi}^2 + r \norm{\varphi}^2_{\omega} \le 0.
\end{equation}

Приведём полезные для доказательства стабилизируемости леммы.

\newtheorem{lemma}{Лемма}

\begin{lemma}\label{util_lemma}
    Система $\left\{ w_j|_{\omega} \right\}^m_1$ линейно независима.
\end{lemma}

\begin{proof}
    % Пусть $D$ - оператор дифференцирования. Он является линейным отображением(преобразованием) $\mathbb{R}^m$ в $\mathbb{R}^m$. Рассмотрим следующее равенство :\\
    Cистема $\left\{ w_j|_{\omega} \right\}^m_1$ линейно независима, если 
    тождество вида
    
    \begin{equation}\label{sum_func}
        \sum \limits_{j = 1}^m{c_j w_j(x)} = 0, \ x \in \omega.
    \end{equation}
    выполняется только при $c_1 = c_2 = ... = c_m = 0.$\\

    Пусть $D$ - оператор дифференцирования, действующий в пространстве бесконечно 
    дифференцируемых функций.
    Заметим, что $D^2 w_j = -(\pi k)^2 w_j$. Подействуем  на \eqref{sum_func} 
    оператором $D^{2l}$ раз

    \begin{equation*}
        c_m (\pi m)^{2l} w_m + \sum \limits_{j = 1}^{m - 1}{c_j (\pi j)^{2l}
        w_j(x)} = 0.
    \end{equation*}

    Разделим на $(\pi m)^{2l}$. Тогда

    \begin{equation*}
        c_m w_m + \sum \limits_{j = 1}^{m - 1}{c_j \left(\frac{j}{m}\right)^{2l}
        w_j(x)}.
    \end{equation*}
    Заметим, что при $l \rightarrow +\infty$, правое слагаемое стремится к 0.\\
    Следовательно,

    \begin{equation*}
        c_m w_m = 0.
    \end{equation*}

    Функция $w_m(x) = \sin{(\pi m x)}$ не может принимать нулевые значения на 
    целом интервале, следовательно $c_m = 0$. Проделав те же самые рассуждения 
    для $\sum \limits_{j = 1}^{m - 1}{c_j w_j(x)}$, получаем, что

    \begin{equation*}
        c_1 = c_2 = ... = c_m = 0.
    \end{equation*}

\end{proof}

\par
\vspace{2ex}

\begin{lemma}\label{main_lemma}
    \begin{equation}
        \gamma = \inf{ \left\{ \norm{z}^2_{\omega} : z = P_mu,\enskip u \in H, 
        \enskip \norm{z} = 1 \ \right\} } > 0.
    \end{equation}
    Здесь $\norm{z}^2_{\omega} = \int\limits_{\omega}{z^2dx}$.
\end{lemma}

\begin{proof}

    Рассмотрим функцию $f$, определённую на единичной сфере в $\mathbb{R}^m$

    \begin{equation}
       f(c_1, c_2, ..., c_m) = \int \limits_{\omega} {(\sum\limits_1^m {c_jw_j})^2}
       \text{, где } c_1^2 + c_2^2 + ... + c^2_m = 1.
    \end{equation}
    По теореме Вейерштрасса, существует функция 
    $z_0 = \sum\limits_1^m {c_j^0 w_j} \in H_m$, такая, что

    \begin{equation}
       \gamma = f(c_1^0, c_2^0, ..., c_m^0) = \inf \limits_{c_1^2 + .. + c_m^2 =
       1} {f}.
    \end{equation}

    Заметим, что из линейной независимости 
    $\left\{ \omega_j \right\}^m_1$(по лемме \ref{util_lemma}) следует 
    положительность $\gamma$.

\end{proof}

\par
\vspace{2ex}

Выберем число $m \in \mathbb{N}$ так, что

\begin{equation}
    q = [(\pi(m + 1))^2 - \alpha - 1] > 0.
\end{equation}
На основании леммы \ref{main_lemma} справедлива следующая оценка
\begin{equation*}
    \frac{1}{2} \frac{d}{dt} \norm{\varphi}^2 + (\pi^2 - \alpha + r\gamma) 
    \norm{\varphi}^2 \le 0.
\end{equation*}

Далее выберем число $r = r_m > 0$ так, чтобы
$\beta = \pi^2 - \alpha + r\gamma > 0$.
Умножим обе части неравенства на $2e^{2\beta t}$, проинтегрируем по $t$ и 
получим следующую важную оценку

\begin{equation}\label{phi_mark}
    \norm{\varphi}^2 \le \norm{\varphi_0}^2 e^{-2\beta t},
\end{equation}
здесь $\varphi_0 = P_m u_0$.
\vspace{2em}

Проведём аналогичные выкладки с $\psi = Q_m u$. Домножим обе части 
\eqref{control} скалярно на $\psi$. Учтем, что

\begin{gather*}
    (u, \psi) = \norm{\psi}^2,\\*
    (u_t, \psi) = (\psi_t, \psi) = \frac{1}{2} \frac{d}{dt} \norm{\psi}^2,\\*
    (u_{xx}, \psi) = (\psi_{xx}, \psi) = -(\psi_x, \psi_x) = -
    \norm{\psi_x}^2,\\*
    (\chi_{\omega}\varphi, \psi) = \int_{\omega}{\varphi \psi} \le
    \norm{\varphi} \norm{\psi}.
\end{gather*}
С помощью неравенства Пуанкаре-Фридрихса-Стеклова получаем оценку
\begin{equation}
    \norm{\psi}^2 \le (\frac{1}{\pi(m + 1)})^2 \norm{\psi_x}^2.
\end{equation}

Тогда

\begin{equation}\label{v2}
    \frac{1}{2} \frac{d}{dt} \norm{\psi}^2 + (\pi(m + 1))^2 \norm{\psi}^2 - 
    \alpha \norm{\psi}^2 \le r \norm{\varphi} \norm{\psi}.
\end{equation}

Рассмотрим подробнее \eqref{v2}.\\
Для оценки произведения норм, воспользуемся неравенством Юнга

\begin{equation*}
    \norm{\varphi} \norm{\psi} \le (\frac{\varepsilon \norm{\psi}^2}{2} + 
    \frac{\norm{\varphi}^2}{2 \varepsilon}),
\end{equation*}
здесь $\varepsilon = \frac{2}{r}$. Из \eqref{v2} получаем неравенства

\begin{equation*}
    \frac{1}{2} \frac{d}{dt} \norm{\psi}^2 + (\pi(m + 1))^2 \norm{\psi}^2 - 
    \alpha \norm{\psi}^2 \le r (\frac{\varepsilon \norm{\psi}^2}{2} + 
    \frac{\norm{\varphi}^2}{2 \varepsilon}),
\end{equation*}
\begin{equation*}
    \frac{1}{2} \frac{d}{dt} \norm{\psi}^2  + [(\pi(m + 1))^2 - \alpha - 1] 
    \norm{\psi}^2 \le \frac{r^2}{4}\norm{\varphi}^2 \le 
    \frac{r^2}{4}\norm{\varphi_0}^2 e^{-2\beta t},
\end{equation*}
\begin{equation}
    \frac{1}{2} \frac{d}{dt} \norm{\psi}^2 + q\norm{\psi}^2 \le 
    \frac{r^2}{4}\norm{\varphi_0}^2 e^{-2\beta t}.
\end{equation}

Напомним, что $q > 0$ за счет выбора числа гармоник в операторе $P_m$.
Домножим обе части неравенства на $2e^{2qt}$ и проинтегрируем
\begin{gather*}
    \int\limits_0^t{d(e^{2q\tau}} \norm{\psi}^2) \le 
    \frac{r^2}{2} \norm{\varphi_0}^2 \frac{1}{2q - 2\beta} 
    \int\limits_0^t {e^{(2q -2\beta) \tau} d[(2q -2\beta) \tau]},\\*
    e^{2qt} \norm{\psi}^2 - \norm{\psi_0}^2 \le \frac{r^2}{2}
    \norm{\varphi_0}^2 \frac{1}{2q - 2\beta} e^{(2q -2\beta)t}.
\end{gather*}

Следовательно, 

\begin{equation}\label{psi_mark}
    \norm{\psi}^2 \le \norm{\psi_0}^2 e^{-2qt} + \frac{r^2}{2} 
    \norm{\varphi_0}^2 \frac{1}{2(q - \beta)} e^{-2\beta t}.
\end{equation}

Известно, что $u = \varphi + \psi$. Воспользуемся оценками норм $\psi$ и 
$\varphi$ [\eqref{psi_mark}, \eqref{phi_mark}] и получим

\begin{gather*}
    \norm{u}^2 = \norm{\varphi}^2 + \norm{\psi}^2 \le \norm{\varphi_0}^2 
    e^{-2\beta t} + \norm{\psi_0}^2 e^{-2qt} + \frac{r^2}{2} \norm{\varphi_0}^2 
    \frac{1}{2(q - \beta)} e^{-2\beta t}.
\end{gather*}

Окончально оценку стабилизации можно записать в виде

\begin{equation}
    \norm{u}^2 \le \norm{u_0}^2 \left( e^{-2\beta t} + e^{-2qt} + 
    \frac{r^2 e^{-2\beta t}}{4(q - \beta)} \right).
\end{equation}

\subsection{Численная реализация алгоритма}
\vspace{1em}

В настоящем параграфе приведена численная реализация предложенного алгоритма 
стабилизации.\\

Рассмотрим задачу с стабилизирующим оператором

\begin{equation}\label{sys}
    u_t - u_{xx} - \alpha u = -r\chi_{\omega}P_m u, \ 0 < x < 1, \quad t > 0
\end{equation}

К уравнению \eqref{sys} добавим начальное и граничные условия

\begin{gather}\label{s_control}
    u(0, t) = u(1, t) = 0 \\*
    u(x, 0) = u_{0}(x) \in H \nonumber
\end{gather}

Для \eqref{sys} запишем разностную схему

\begin{equation}\label{scheme}
    \frac{u^{j + 1}_i - u^j_i}{\tau} - \frac{u_{i + 1}^{j + 1} - 
    2u_{i}^{j + 1} + u_{i - 1}^{j + 1}}{h^2} - \alpha u_{i}^{j + 1} + 
    r\chi_{\omega}P_m u^j_i = 0
\end{equation}

Запишем аппроксимацию начального и граничных условий

\begin{gather}
    u_i^0 = u_0(x_i) \\*
    u_1^{j+1} = u_N^{j+1} = 0 \nonumber
\end{gather}

Вспомним, что оператор проектирования имеет вид

\begin{gather*}
    P_m u = \sum \limits_{j=1}^{m} {(u, w_j) w_j} = 
    \sqrt{2} (\sum \limits_{j=1}^{m} {C_k \sin{(\pi k x)}}), \ \text{где }
    C_k = \sqrt{2} \int\limits_0^1{u(s)\sin{(\pi k s)} ds}
\end{gather*}

Заметим, что $C_k$ - это интеграл от быстро осциллирующей функции вида

\begin{equation}
    \int\limits_a^b{f(x) e^{i\omega x} dx} \approx \int\limits_a^b{L_3(x) e^{i\omega x} dx}
\end{equation}

Поскольку, функция $f$ является гладкой, то на $[a, b]$ она легко приближается 
с известной погрешностью интерполяционными многочленами. Пусть для 
определенности, это интерполяционный многочлен в форме Лагранжа

\begin{equation}
    L_3(x) = P_1(x)f(x_1) + P_2(x)f(x_2) + P_3(x)f(x_3)
\end{equation}

построенный по узлам $x_1 = a$, $x_2 = \frac{a + b}{2}$, $x_3 = b$. $P_i$ - 
многочлены второй степени, не зависящие от функции $f$. Данный метод 
приближенного интегрирования называется формулой Филона. Именно этим 
способом и будем аппроксимировать оператор $P_m$.\\

Для решения данной схемы \eqref{scheme} воспользуемся методом прогонки.

\subsection{Примеры численного моделирования стабилизации неустойчивых
решений уравнения теплопроводности}

\vspace{1em}

\newtheorem{exmp_st}{Пример}

\begin{exmp_st}
\end{exmp_st}

В качестве начальных условий возмем $u_0 = sin(\pi x)$. Продемонстрируем 
стабилизацию системы при $\alpha = \pi^2 + 0.1$. Фиксируем $\omega = [0, 0.2]$. 
Необходимо подобрать параметры $m$, $r_m$ так, чтобы $\beta > 0$ и $q > 0$. 
Рассмотрим подробнее $q = [(\pi(m + 1))^2 - \alpha - 1]$. При заданном 
$\alpha = \pi^2 + 0.1$, достаточно взять $m = 2$ для выполнения неравенства. 
Параметр $r$ придется подобрать так, чтобы решение стремилось к нулю.


\begin{figure}[H]
    \centering
    \includegraphics[width=4in]{par_re_pi01}
    \caption{Управление $m = 2,\; r = 8$}
    \label{fig:test2}
\end{figure}

\begin{exmp_st}
\end{exmp_st}

Пусть $u(x, 0) = x(1 - x)$ - начальное условие . Заведомо выберем параметр 
$\alpha = \pi^2 + 3$ большим. Зафиксируем $\omega = (0, 0.4)$. 
Необходимо подобрать $m$, таким чтобы $q > 0$. При $m \ge 2$ условие выполняется, 
поэтому мы фиксируем $m = 2$. На рис.4 показано, как быстро растет решение 
задачи \eqref{sys} - \eqref{s_control} при небольшом увеличении $\alpha$. 
Стабилизация этой системы представлена на рисунке 5

\begin{figure}[H]
    \centering
    \includegraphics[width=4in]{par_re_pi3}
    \caption{Управление $m = 2,\; r = 15$}
    \label{fig:test2}
\end{figure}

\section{Стабилизация неустойчивых стационарных решений уравнения Бюргерса}
\vspace{1em}

\subsection{Постановка задачи}

Рассмотрим уравнение Бюргерса с вязкостью $\nu > 0$ на интервале $\Omega = (0,
1) \subset \mathbb{R}$

\begin{equation}\label{burger}
    u_t - \nu u_{xx} + u_x u = f + y, \ u|_{\Gamma} = u_b, \quad t > 0
\end{equation}

Функция $f = f(x)$, $x \in \Omega$ является заданной, а функция $y = y(x, u)$
рассматривается как управление, носитель которого при фиксированном $t > 0$
содержиться в $\bar{\omega}$, где $\bar{\omega}$ - заданная подобласть $\Omega$.
Через $\Gamma = \{0, 1\}$ обозначена граница $\Omega$, $u_b \in \mathbb{R}^2$\\

Пусть $U$ - стационарное решение \eqref{burger}, то есть

\begin{equation}\label{stationary_sol}
    -\nu U_{xx} + U U_x = f, \ U|_{\Gamma} = u_b
\end{equation}

и $U$ является неустойчивой особой точкой динамической системы, порождаемой
эволюционным уравнение \eqref{burger} в фазовом пространстве $H = L^2(\Omega)$.
Задача стабилизации состоит в следующем:\\

\textit{Для заданного} $\sigma > 0$ 
\textit{требуется найти оператор управления с обратнойсвязью} 
$y = \Lambda(u - U) : H \to H$ \textit{такой, что} $\mathbf{supp}y(\cdot,t) \subset 
\bar{\omega}$ \textit{и решение замкнутой системы}

\begin{equation}
    u_t - \nu u_xx + u u_xx = f + \Lambda(u - U), \ u|_{\Gamma} = u_b,
    \ t > 0, \quad u|_{t=0} = u_0
\end{equation}

\textit{сходится к} $U$ \textit{с заданной скоростью} $\sigma$

\begin{equation}
    \norm{u(t) - U}_{L^2(\Omega)} \le C e^{-\sigma t} \ \text{при } t
    \to +\infty
\end{equation}

\textit{если величина} $\norm{u_0 - U}_{L^2(\Omega)}$ \textit{достаточно мала}\\

Пусть $\varphi = u - U$, тогда

\begin{gather}\label{fluct}
    \varphi_t - \nu \varphi_{xx} + \varphi U_x + (\varphi + U)\varphi_x =
    \Lambda(\varphi)\\* 
    \varphi|_{\Gamma} = 0, \ t > 0\\*
    \varphi|_{t = 0} = \varphi_0 = u_0 - U
\end{gather}

Требуется чтобы $\norm{u(t) - U}_{L^2(\Omega)} \le C e^{-\sigma t}$ при $t \to
+\infty$, если мала норма $\norm{\varphi_0}_{L^2(\Omega)}$

\subsection{Неустойчивость стационарных решений shock-like}

Рассмотрим семейство стационарных решений shock-like

\begin{equation}\label{shock_like}
    U(x) = -2\sigma\tanh{(\sigma(x - \frac{1}{2}))}, \ \text{где } \sigma \ge 0
\end{equation}

\begin{figure}[H]
    \centering
    \includegraphics[width=4in]{fig1}
    \caption{$U(x)$ при разных $\sigma$}
\end{figure}

Для изучения устойчивости системы \eqref{fluct}, мы линеаризуем её

\begin{gather}\label{linearized}
    \theta_t = \theta_{xx} + 2 \sigma (\tanh(\sigma(x - \frac{1}{2}))\theta)_x \\*
    \theta(0, t) = \theta(1, t) = 0,
\end{gather}

где $\theta(x, t)$ - решение уравнения \eqref{linearized}, которое является
линеаризацией \eqref{fluct}. Заметим что \eqref{linearized} является  уравнение 
конвенкции-диффузии-реакции. Для простоты изучения устойчивости, мы избавимся 
от конвекционого члена используя преобразование 
$\zeta(x, t) = G(x)\theta(x, t)$, где 

\begin{equation}
    G(x) = \frac{\cosh(\sigma(x - \frac{1}{2}))}{\cosh(\frac{\sigma}{2})}
\end{equation} 

Имеем 

\begin{gather} \label{transf_linear}
    \zeta_t = \zeta_{xx} + \sigma^2 \left( \frac{2}{\cosh^2(\sigma(x - \frac{1}{2}))} - 1 \right) \zeta \\* 
    \zeta(0) = \zeta(1) = 0 
\end{gather}

Для $\sigma = 0$ система нейтрально устойчива. Для $\sigma > 0$, член 
$\left(\frac{2}{\cosh^2(\sigma(x - \frac{1}{2}))} - 1 \right)$  в 
\eqref{transf_linear} также является неустойчивым в окрестности 
$x = \frac{1}{2}$ (Рис 2.), т.е. положительность этого члена ведет к
неустойчивости системы


\begin{figure}[H]
    \centering
    \includegraphics[width=4in]{fig2}
    \caption{Значение реакционного члена в \eqref{transf_linear}}
\end{figure}

\subsection{Примеры неустойчивых стационарных решений}

Приведем численные примеры, демонстрирующие зависимость устойчивости
стационарного решения \eqref{shock_like} от параметра $\sigma > 0$ и от
"возмущающих" гармоник $\theta_0$

\newtheorem{exmp_bur}{Пример}
\begin{exmp_bur}
\end{exmp_bur}

Начальное условие $\theta_0(x) = \frac{\sin(\pi x)}{G(x)}$, параметр $\sigma$ 
возмем равный 15. На рисунке 3 показано неустойчивое поведение системы
\eqref{fluct}

\begin{figure}[H]
    \centering
    \includegraphics[width=4in]{ex_s15}
    \caption{Без управления}
\end{figure}

\begin{exmp_bur}
\end{exmp_bur}
Теперь начальное условия $\theta_0(x) = \frac{x^2}{G(x)}$. Параметр $\sigma = 15$. 

\begin{figure}[H]
    \centering
    \includegraphics[width=4in]{ex_x2_s15}
    \caption{Без управления}
\end{figure}

\section{Введение}

Пусть

\begin{equation}
    \Omega = (0, 1), \quad \omega \subset \Omega
\end{equation}

Неустойчивое уравнение Бюргерса

\begin{gather}
    y_t - \nu y_{xx} + yy_x = f \\*
    y|_{\partial \Omega} = y_{bi} \\*
    y|_{t=0} = y_0
\end{gather}

Стационарное уравнение Бюргерса

\begin{gather}
    y_{sxx} + y_s y_{sx} = f_s \\*
    y|_{s \partial \Omega} = y_{bi}
\end{gather}

\subsection{Задача стабилизации}
Найти оператор управления с обратной связью $\Lambda$ такой, что решение задачи
(1), где $f = f_s + \Lambda (y - y_s)$ экспоненциально сходится к $y_s$

Обозначим за $\varphi = y - y_s$, тогда

\begin{gather}
    \varphi_t - \nu \varphi_{xx} + y_s \varphi_x + \varphi y_{sx} + \varphi
    \varphi_x = \Lambda(\varphi) \\*
    \varphi_0 = y_0 - y_s \\*
    \varphi|_{\partial \Omega} = 0
\end{gather}

\subsection{Формализация}
Здесь и далее
\begin{equation}
    H = L^2(\Omega), \quad V = H^1_0(\Omega), \quad \norm{z}^2 = \int_{\Omega}{z^2dx}
\end{equation}

\begin{equation}
    V \subset H = H^{'} \subset V^{'} = H^{-1}(\Omega), (y, z) =
    \int_{\Omega}{yzdx}
\end{equation}

\begin{equation}
    A: V \rightarrow V^{'}, (Ay, z) = (y_x, z_x)
\end{equation}

\begin{equation}
    B: V \times V \rightarrow V^{'}, \quad (B(y, z), w) = (yz_x, w)
\end{equation}

\begin{equation}
    A w_j = \lambda_j w_j, \quad j = 1, 2.. \quad \lambda_j \rightarrow +\infty
\end{equation}

\begin{equation}
    \lambda_j = (\pi j)^2, \quad w_j(x) = \sqrt{2}\sin{(\pi j x)}
\end{equation}

Заметим что базис ортонормирован в $H$

\begin{equation}
    (w_j, w_k) = \delta_{jk}
\end{equation}

Введем следующие обозначения

\begin{equation}
    H_m = span{w_1, ..., w_m}, \quad P = P_m: H \rightarrow H_m, \quad Q=Q_m = I
    - P
\end{equation}

\subsubsection{Неравенства}

\begin{equation}
    \lambda_1 \norm{v}^2 \le \norm{v_x}^2, \quad \norm{v}^4_{L^4} \le \norm{v}^2
    \norm{v_x}^2
\end{equation}

\begin{equation}
    |(B(y, z), w)| \le \norm{y}_{L^4} \norm{z_x} \norm{w}_{L^4} \le
    \norm{y}^{1/2} \norm{y_x}^{1/2} \norm{z_x} \norm{w}^{1/2} \norm{w_x}^{1/2}
\end{equation}

\begin{equation}
    \norm{w}^2 \le \lambda^{-1}_{m + 1} \norm{w_x}^2 \forall w \in V \cap
    H_m^{\bot}
\end{equation}

\begin{equation}
    |v(x)| \le \norm{v_x}, \quad |(B(y, z), w)| \le \norm{y_x} \norm{z_x}
    \norm{w} \text{или} \norm{y} \norm{z_x} \norm{w_x}
\end{equation}

Оператор локально распределенного управления $\Lambda: H \rightarrow H$, \\
\begin{equation}
    \Lambda(\varphi) = -r \xi_{\omega}P\varphi
\end{equation}


Здесь \\
\begin{gather*}
    \begin{matrix}
        r & =
        & \left\{
        \begin{matrix}
            r_0, & 2 \norm{Q\varphi} \le \norm{P\varphi}, \quad r_0 > 0, \\
            0, & \mbox{иначе. }
        \end{matrix} \right.
    \end{matrix}, \quad
    \begin{matrix}
        \chi_{\omega}(x) & =
        & \left\{
        \begin{matrix}
            0, & \mbox{если } x \notin \omega, \\
            1, & \mbox{иначе. }
        \end{matrix} \right.
    \end{matrix}
\end{gather*}

Перепишем задачу в следующем виде

\begin{gather}
    \varphi' + \nu A\varphi + B(\varphi, y_s) + B(y_s + \varphi, \varphi) =
    \Gamma(\varphi), \quad t > 0 \\*
    \varphi(0) = \varphi_0 = y_0 - y_s
\end{gather}

\subsection{Разрешимость задачи на конечном интервале времени}

\newtheorem{theorem}{Теорема}

\begin{theorem}
    Пусть $y_s \in H^1, \quad \varphi_0 \in H, \quad T > 0$. Тогда $\exists$ решение
    задачи $\varphi \in C([0, T], H) \cap L^2(0, T, V), \varphi' \in L^2(0, T, V')$
\end{theorem}

\begin{remark}
    Пусть $h(\lambda)$ единичная функция Хевисайда, \\
    \begin{gather*}
        \begin{matrix}
            h(\lambda) & =
            & \left\{
            \begin{matrix}
                1, & \lambda \ge 0 \\
                0, & \lambda < 0
            \end{matrix} \right.
        \end{matrix}
    \end{gather*}

    , тогда оператор управления примит следующий вид

    \begin{equation}
        \Lambda(\varphi) = -r_0 h(-2\norm{Q\varphi} +
        \norm{P\varphi})\chi_{\omega}P\varphi
    \end{equation}

\end{remark}


\begin{proof}
    Пусть $\Omega > 0$. Выберим параметры $\sigma$, $r_0$, $m$ такие, что
    $\norm{g(t)}$ ограничена на $[0, +\infty)$. Здесь $g(t) = \varphi(t)e^{\sigma t}$
    \begin{gather*}
        g' + \nu Ag - \sigma g + B(g, y_s) + B(y_s + ge^{-\sigma t}, g) = -r_0 +
        h(\norm{Pg} - 2 \norm{Qg}) + \chi_{\omega}Pg\\*
        g(0) = \varphi_0
    \end{gather*}
    Пусть $p = Pg$, $q = Qg$. Рассмотрим интервал $(t_0, t_1)$, где $\norm{p(t)}
    \le 2 \norm{q(t)}$:
    \begin{gather*}
        g' + \nu Ag - \sigma g + B(g, y_s) + B(y_s + ge^{-\sigma t}, g) = 0, \quad
        t_0 < t < t_1\\
        \frac{1}{2}\frac{d}{dt}\norm{g}^2 + \norm{g_x}^2 - \sigma \norm{g}^2 +
        \frac{1}{2}(y_{sx}, g^2) = 0
    \end{gather*}

    \begin{gather*}
        \norm{g}^2 = \norm{p}^2 + \norm{q}^2 < 5\norm{q}^2 < 5\lambda^{-1}_{m +
        1}\norm{q_x} \le 5 \lambda^{-1}_{m + 1}\norm{g_x}^2
    \end{gather*}

    Пусть 
    \begin{equation}
        C_1 = \sigma + \frac{1}{2}max|y_{sx}|, \quad 5C_1\lambda^{-1}_{m + 1} <
        \frac{\nu}{2}
    \end{equation}

    Тогда отсюда 
    \begin{equation}
        \frac{d}{dt}\norm{g}^2 + \nu \norm{g_x} < 0
    \end{equation}

    Тогда следует

    \begin{equation}
        \norm{g(t_1)} < \norm{g(t_0)}
    \end{equation}

    Далее 
    \begin{gather*}
        \frac{1}{2}\frac{d}{dt}\norm{q}^2 + \nu \norm{q_x}^2 - \sigma \norm{q}^2 +
        ((y_s g)_x, q) + \frac{1}{2}e^{-\sigma t}((g^2)_x, q) = 0\\
        \norm{g_x}^2 = \norm{p_x}^2 + \norm{q_x}^2 \le \lambda_{m} \norm{p}^2 +
        \norm{q_x}^2 \le 5\norm{q_x}^2\\
        |(y_s g)_x, q)| \le C_0 \norm{g} \norm{q_x} < 5 C_0 \lambda^{-1}_{m + 1}
        \norm{q_x}^2, \quad C_0 = max|y_s|\\
        |((g^2)_x, q)| = |(g^2, q_x)| \le \norm{g} \norm{g_x} \norm{q_x}
    \end{gather*}
    
    Пусть теперь $\norm{p(t)} \ge 2 \norm{q(t)}, \quad t \in (t_0, t_1)$\\
    \begin{gather*}
        g' + \nu Ag - \sigma g + (y_s g)_x + \frac{1}{2} e^{-\sigma t} (g^2)_x +
        r_0 \chi_{\omega} P = 0
    \end{gather*}

    Заметим что
    \begin{gather*}
        \gamma |(p, q)_{L^2(\omega)}| \le \norm{p} \norm{q} \le \frac{1}{2}
        \norm{p}^2 \text{следует, что} \\
        \norm{p}^2_{L^2(\omega)} + \gamma (p,
        q)_{L^2(\omega)} \ge \frac{\gamma}{2}\norm{p}^2
    \end{gather*}
    
    \begin{gather*}
        \frac{1}{2} \frac{d}{dt} [ \norm{p}^2 + \nu [ \norm{p_x}^2 - \sigma [
        \norm{p}^2 + \gamma \norm{q}^2] ]  + ( (y_s g)_x, p + \gamma q) + \gamma
    \norm{q} \\* + \gamma\norm{q_x}^2] + \frac{1}{2} e^{-\sigma t} ( (g^2)_x, p +
    \sigma q) + \frac{r_0 \gamma}{2} \norm{p}^2 \le 0\\
    |( (y_s g)_x, p)| \le C_0 \norm{p_x} \norm{g} \le \frac{3}{2} C_0 \norm{p}
    \norm{p_x} \le \frac{\nu}{2} \norm{p_x}^2 + \frac{1}{2\nu}(\frac{3}{2}C_0)^2
    \norm{p}^2
    \end{gather*}

    \begin{equation}
        \norm{g} \le \norm{p} + \norm{q} \le \frac{3}{2} \norm{p}
    \end{equation}

    \begin{gather*}
        |((y_s g)_x, q)| \le C_0 \norm{q_x} \norm{g} \le \frac{3}{2} C_0
        \norm{q_x} \norm{p} \le \frac{\nu}{2} \norm{q_x}^2 + \frac{9C_0^2}{8\nu}
        \norm{p}^2
    \end{gather*}

    \begin{align*}
        \frac{d}{dt}(\norm{p}^2 + \gamma \norm{g}^2) + \nu(\norm{p_x}^2 + \gamma
        \norm{q_x}^2) + (r_0\gamma - (2 + \gamma)\sigma - (1 +
        \gamma)\frac{9C_0^2}{4\nu})\norm{p}^2 \\* \le |(g^2, p_x + \gamma q_x)|
    \end{align*}

    \begin{equation}
        r_0 > C_2 = (1 + \frac{2}{\gamma})\sigma + (1 +
        \frac{1}{\gamma})\frac{9C_0^2}{4\nu}
    \end{equation}

    Оценим правую часть 

    \begin{gather*}
        |(g^2, p_x + \gamma q_x)| = (1 - \gamma)|(g^2, p_x)| = (1 - \gamma)|(p^2
        + 2pq + q^2, p_x)| \\*
        = (1 - \gamma)|(q^2, p_x) - (p^2, q_x)| \le \norm{p_x} \norm{q_x}
        (\norm{p} + \norm{q}) \\* \le
        \frac{1}{2\sqrt{\gamma}}(\norm{p_x}^2 + \gamma \norm{q_x}^2)
        \frac{2}{\sqrt{\gamma}}\sqrt{\norm{p}^2 + \gamma \norm{q}^2}
    \end{gather*}
    т.к. $\norm{p} + \norm{q} \le \frac{2}{\sqrt{\gamma}}\sqrt{\norm{p}^2 +
        \gamma \norm{q}^2}$ отсюда следует, что

    \begin{equation}
        \frac{d}{dt}(\norm{p}^2 + \gamma \gamma{q}^2) + (\norm{p_x}^2 + \gamma
        \norm{q_x}^2) (\nu - \frac{1}{\nu} \sqrt{\norm{p}^2 + \gamma \norm{q}^2}
        \le 0
    \end{equation}

    
\end{proof}

\subsection{Примеры численного моделирования стабилизации неустойчивых решений}
\vspace{1em}

\subsubsection{Стабилизация неустойчивых решений}

\newtheorem{exmp_stbur}{Пример}

\begin{exmp_stbur}
\end{exmp_stbur}

Пусть $\theta_0 = \frac{sin(\pi x)}{G(x)}$ - начальное условие. Рассмотрим
стационарное решение уравнения Бюргерса с параметром $\tau$ возмем равный 15. 
В предыдущем параграфе показано неустойчивое поведение системы без управления. 
Cтабилизация указанного решения за счет управления с параметрами 
$\omega = (0, 0.4), \; r = 30, \; m = 4$ предоставлена ниже


\begin{figure}[H]
  \centering
  \includegraphics[width=3.5in]{re_s15}
  \caption{Управление с $\omega = (0, 0.4), \; m = 4, \; r = 30$}
  \label{fig:test2}
\end{figure}


\begin{exmp_stbur}
\end{exmp_stbur}
Пусть теперь начальное условие системы $\theta_0(x) = \frac{x^2}{G(x)}$.
Впспомним, что чем больше параметр $\tau > 0$, тем сильнее он влияет на
неустойчивость нашей системы \eqref{fluct}. Зафиксируем следующие параметры
стабилизирующего оператора управления $\omega = (0, 0.4), \ r = 30$, а параметр 
$\tau$  возмем равный $15$. Неустойчивость этой системы предемонстирована в
предыдущем параграфе (рис 2). Ниже представлен процесс стабилизации

\begin{figure}[H]
 \centering
  \includegraphics[width=3.5in]{re_x2_s15}
  \caption{$\omega = (0, 0.4), \; r = 30, \; m = 3$}
  \label{fig:test2}
\end{figure}

\subsubsection{Анализ параметров оператора стабилизации}

Рассмотрим ....



\backmatter

%\section*{Заключение}
%\addcontentsline{toc}{section}{Заключение}
\Conclusion
\vspace{2em}

В представленной ВКР предложен простой метод стабилизации неустойчивых
стационарных решений параболических уравнений. Особенность предложенного
алгоритма заключается в локальности носителя управления и конечномерности образа
оператора управления. В работе получены следующие результаты:

\begin{itemize}

\item{Разработан алгоритм реализации стабилизирующего
управления для линейного уравнения теплопроводности и нелинейного уравнения
Бюргерса;}

\item{Проведён теоретический анализ устойчивости неустойчивых параболических
систем;}

\item{Реализован алгоритм стабилизации;}

\item{Приведены примеры неустойчивых стационарных решений, примеры численного 
моделирования стабилизации неустойчивых систем.}

\end{itemize}
%разработан алгоритм реализации стабилизирующего
%управления для линейного уравнения теплопроводности и нелинейного уравнения
%Бюргерса, теоретическое обоснование методов стабилизации, в том числе написание программы для решения разностных схем и
%реализация метода Филона для интегралов от быстро осциллирующих функций.
%Приведены примеры численного моделирования неустойчивых стационарных решений
%линейных параболических уравнений на примере уравнения теплопроводности и нелинейных для неустойчивых
%стационарных решений уравнения Бюргерса типа "shock-like", и применение
%предложенного алгоритма стабилизации.

\section{Список литературы}
\vspace{2em}

\begingroup
\renewcommand{\section}[2]{}%
%\renewcommand{\chapter}[2]{}% for other classes
\begin{thebibliography}{}
\bibitem{}
	А.Ю. Чеботарёв, Конечномерная стабилизация с заданой скоростью систем типа Навье-Стокса. Дальневосточный мат. журнал 2010. Том 10 $№2$ с 200-205

\bibitem{}
  Miroslav Krstic, Lionel Magnis, Rafael Vazquez, Nonlinear Control of the Burgers PDE—Part I: Full-State Stabilization.\ June 11-13, 2008

\bibitem{}
  Alexei Kourbatov, Lyapunov Exponents for Burgers’ Equation. Sbornik Nauchnykh Trudov MFTI. Moscow, 1992

\bibitem{oe04}
    M. Krstic, A. Smyshlyaev, Boundary Control of PDEs.\ 2008
\bibitem{}
	M. Krstic, A. Smyshlyaev, Boundary Control of PDEs (short course).\ 2006
\bibitem{}
    А. В. Фурсиков. Стабилизируемость квазилинейного параболического уравнения с помощью граничного управления с обратной связью. Матем. сб., 192:4 (.\ 2001)
\bibitem{}
	O.А. Ладыженская, Краевые задачи математической физики.\ 1973
\bibitem{}
	А.А. Самарский, А.В. Гулин, Численные методы. общий базовый курс.\ 1989
\bibitem{}
	Метод прогонки, неявная схема [Электронный ресурс] : \url{http://ikt.muctr.ru/html2/4/lek4_2_1.html}
\bibitem{}
	Формула Филона [Электронный ресурс] : \url{http://window.edu.ru/resource/886/19886/files/rsu177.pdf}

\end{thebibliography}
\endgroup



\end{document} 
