%\chapter*{Введение}
%\addcontentsline{toc}{section}{Введение}
%\vspace{1em}

\Introduction

В реальных физических процессах неизбежно возникают непредусмотренные 
флуктуации, и поэтому возникает необходимость разработки методов построения 
управлений, способных реагировать на непредусмотренные возмущения и подавлять 
их \cite{Furs}. Проблемы стабилизации систем параболического типа привлекают внимание 
специалистов в силу прикладной значимости данной системы \cite{Chebotarev}. 
Управления такого типа называются \emph{управлениями с обратной связью} \cite{KS}.

Вопрос о стабилизируемости различных эволюционных уравнений в частных 
производных с помощью управлений исследовался многими авторами, среди которых 
A.Kwiecinka \cite{KWCK}, Barbu V. \cite{Barbu}, M. Kristic
\cite{KMV, KS}, Фурсиков А.В. \cite{Furs}, Чеботарёв А.Ю. 
\cite{Chebotarev, ChebotarevBS, ChebotarevMGT} и разработаны методы управления, 
такие как : стабилизация Ляпунова, метод backstepping, управление с обратной
связью. Существующие методы стабилизации довольно громоздкие и требуют значительных 
вычислительных затрат, тогда как алгоритм рассматриваемый в ВКР прост в реализации.

Представленная ВКР посвящена изучению алгоритмам стабилизации для неустойчивых
по Ляпунову линейных и нелинейных параболических систем. В работе
рассматривается алгоритм стабилизация эволюционных систем с постоянными коэффициентами 
конечномерным локальным управлением с обратной связью. Указанный
метод позволяет провести стабилизацию с любой наперёд заданной скоростью $\sigma
> 0$ засчет выбора параметров стабализирующего оператора.

В работе проделаны теоретический анализ устойчивости линейной параболической
системы на примере уравнения теплопроводности, численные примеры иллюстрирующие
неустойчивые решения, примеры численного моделирования стабилизации, теоретическое
обоснование алгоритма стабилизации, предложен метод стабилизации неустойчивых
стационарных решений типа "shock-like" для вязкого уравнения Бюргерса.
