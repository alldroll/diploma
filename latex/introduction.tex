\section*{Введение}
\addcontentsline{toc}{section}{Введение}
\vspace{1em}

В реальных физических процессах неизбежно возникают непредусмотренные 
флуктуации, и поэтому возникает необходимость разработки методов построения 
управлений, способных реагировать на непредусмотренные возмущения и подавлять 
их \cite{Furs}. Проблемы стабилизации систем параболического типа привлекают внимание 
специалистов в силу прикладной значимости данной системы \cite{Chebotarev}. 
Управления такого типа называются \emph{управлениями с обратной связью} \cite{KS}.

Вопрос о стабилизируемости различных эволюционных уравнений в частных 
производных с помощью управлений исследовался многими авторами, среди которых 
A.Kwiecinka \cite{KWCK}, Barbu V. \cite{Barbu}, M. Kristic
\cite{KMV, KS}, Фурсиков А.В. \cite{Furs}, Чеботарёв А.Ю. 
\cite{Chebotarev, ChebotarevBS, ChebotarevMGT} и разработаны методы управления, 
такие как : стабилизация Ляпунова, метод backstepping, управление с обратной связью.

В представленной ВКР рассматривается стабилизация эволюционных систем с 
постоянными коэффициентами конечномерным локальным управлением с обратной связью.
