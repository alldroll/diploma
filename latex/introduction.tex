\section*{Введение}
\addcontentsline{toc}{section}{Введение}
\vspace{1em}

В реальных физических процессах неизбежно возникают непредусмотренные 
флуктуации, и поэтому возникает необходимость разработки методов построения 
управлений, способных реагировать на непредусмотренные возмущения и подавлять 
их. Проблемы стабилизации для систем параболического типа, привлекают внимание 
специалистов в силу прикладной значимости данной системы \cite{Chebotarev}. Управления такого 
типа называются \emph{управлениями с обратной связью} \cite{KS}.

Вопрос о стабилизируемости различных эволюционных уравнений в частных 
производных, с помощью управлений исследовался многими авторами, среди которых 
J.-L. Lions \cite{}, V.Komornik \cite{}, J.-M. Coron \cite{}, M. Kristic
\cite{KMV, KS}, и разработаны методы 
управления, такие как : стабилизация Ляпунова, метод backstepping, управление 
с обратной связью.

В данной дипломной работе рассматривается стабилизация эволюционных систем с 
постоянными коэффициентами конечномерным локальным управлением с обратной связью.
