\section{Стабилизация неустойчивых параболических систем}
\vspace{1em}

\subsection{Анализ устойчивости линейного параболического уравнения}

Рассмотрим параболическое уравнение

\begin{equation}\label{dif_form}
    u_t = u_{xx} + \alpha u, \ 0 < x < 1, \ t > 0.
\end{equation}
с начальным и граничными условиями:
\begin{gather}\label{d_control}
    u(0, t) = u(1, t) = 0, \\*
    u(x, 0) = u_{0}(x) \in L^2(0, 1). \nonumber
\end{gather}

Здесь и далее через $u_t$, $u_x$, $u_{xx}$ .. обозначаются соотвествующие
частные производные функции $u$.\\
Умножим уравнение \eqref{dif_form} на $u$ скалярно в $L^2(0, 1)$

\begin{equation*}
    (u_t, u) = (u_{xx}, u) + \alpha (u, u).
\end{equation*}

Скалярное произведение в $L^2(0, 1)$ определяется как $(u, v) = \int_0^1 uv dx$,
а норма как $\norm{u} = \sqrt{(u, u)}$. Получаем

\begin{equation}\label{int_form}
    \frac{1}{2}\frac{d}{dt}\norm{u}^2 = -\norm{u_x}^2 + \alpha \norm{u}^2.
\end{equation}
C помощью неравенства Пуанкаре–Фридрихса-Стеклова
\begin{equation*}
    \norm{u}^2 \le \frac{1}{\pi^2} \norm{u_x}^2.
\end{equation*}
получаем следующую оценку 
\begin{equation}\label{stable_opr}
    \frac{1}{2}\frac{d}{dt}\norm{u}^2 \le (\alpha - \pi^2)\norm{u}^2.
\end{equation}
Рассмотрим 3 случая.

\begin{enumerate}
    \item $\alpha = \pi^2$. Тогда из \eqref{stable_opr} следует неравенство
        \begin{equation}
            \norm{u}^2 \le \norm{u_0}^2.
        \end{equation}

        Указанное неравенство означает, что нулевое решение задачи
        \eqref{dif_form} - \eqref{d_control} устойчиво по Ляпунову, но не 
        устойчиво ассимптотически.

    \item $\alpha < \pi^2$. Обозначим $\frac{\mu}{2} = -(\alpha - \pi^2)$,\\

        тогда

        \begin{equation}\label{less_pi2}
            \frac{d}{dt}\norm{u}^2 + \mu \norm{u}^2 \le 0.
        \end{equation}

        Домножим обе части \eqref{less_pi2} на $e^{\mu t}$. Тогда, 
        $\frac{d(\norm{u}^2 e^{\mu t})}{dt} \le 0$. Проинтегрируем по $t$ и в 
        итоге получим

        \begin{equation*}
            \norm{u}^2 \le \norm{u_0}^2 e^{-\mu t} = \norm{u_0}^2 e^{2(\alpha -
            \pi^2) t}.
        \end{equation*}

        Данная оценка гарантирует ассимптитическую экспоненциальную устойчивость.

    \item $\alpha > \pi^2$. Решение начально краевой задачи 
        \eqref{dif_form} - \eqref{d_control} имеет вид
        \begin{equation}
            u(x, t) = 2 \isum{a_j e^{(\alpha - \pi^2 j^2)t}\sin{(\pi j x)}},
        \end{equation}

        здесь $a_j = \int_0^1{u_0 \sin{(\pi j s)} ds}$. Первый член суммы 
        $a_1e^{(\alpha - \pi^2j^2)t}\sin(\pi j x)$ указывает на темп роста 
        решения при $t \rightarrow \infty$. Следовательно, система неустойчива.
\end{enumerate}

Для стабилизации системы \eqref{int_form} в случае 3, будем использовать ниже 
описанный метод, предложенный Чеботаревым А.Ю.

%TODO сделать ссылку на мгт.
