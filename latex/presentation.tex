\documentclass{beamer}
\usepackage[T2A]{fontenc}       %поддержка кириллицы
\usepackage[utf8]{inputenc}
\usepackage{graphicx}

\usetheme{Frankfurt}

\title{Стабилизация неустойчивых параболических систем}
\author{Петров Александр}
\date{4 июня 2014 г.}

\graphicspath{{images/}}
\setbeamertemplate{frametitle}[default][left]

\newcommand{\norm}[1]{\left\lVert#1\right\rVert}
\newcommand{\isum}[2][j]{\sum \limits_{#1=1}^{\infty}{#2}}
\newcommand{\operator}[1]{\mathcal{R}{#1}}
\newcommand{\supp}{\mathop{\mathrm{supp}}}

\begin{document}
\setbeamertemplate{caption}[numbered]

\begin{frame}
\titlepage
\end{frame}


\begin{frame}
\frametitle{Введение}
\hspace{5mm} В реальных физических процессах неизбежно возникают непредусмотренные флуктуации и поэтому возникает необходимость разработки методов построения управлений, способных реагировать и подавлять возмущения. Управления такого типа называются \emph{управления с обратной связью}. В данной курсовой работе решались следующие задачи
\begin{itemize}
	\item Проанализирована устойчивость линейной параболической системы;
	\item Рассмотрен метод стабилизации линейной параболической системы с постоянными коэффициентами конечномерным локальным управлением с обратной связью; 
	\item Проделаны численные эксперименты, показывающие эффективность данного метода.
\end{itemize}
\end{frame}


\begin{frame}
\frametitle{Устойчивость параболической системы}

\begin{block}{}
\begin{equation}\label{dif_form}
u_t = u_{xx} + \alpha u, \quad 0 < x < 1, \quad t > 0
\end{equation}
\end{block}
с начальным и граничными условиями:
\begin{block}{}
\begin{gather}\label{d_control}
u(0, t) = u(1, t) = 0, \\*
u(x, 0) = u_{0}(x) \in L^2(0, 1). \nonumber
\end{gather}
\end{block}

\begin{tabular}{@{\textbullet~}l@{\ }p{3in}}
  \bfseries $\alpha = \pi^2$ : & Устойчиво по Ляпунову, но не устойчиво ассимптотически \\
  \bfseries $\alpha < \pi^2$ : & Ассимпотитески устойчиво \\
  \bfseries $\alpha > \pi^2$ : & Неустойчивое нулевое решение
\end{tabular}

\end{frame}

\begin{frame}
\frametitle{Конструкция оператора управления}
Здесь и далее 
\begin{block}{}
\begin{equation}
	H = L_2(0, 1), V = H^1_0(0, 1)
\end{equation}
\end{block}

\hspace{5mm}Пусть $\omega \subset (0, 1)$, такой что $\bar{\omega} \subset (0, 1)$. Задача стабилизации за счет конечномерных локально распределённых в $\omega$ управлений заключается в построении оператора $\mathcal{R} : H \rightarrow H$ такого, что
\begin{enumerate}
\item $\forall z \in H \quad \supp \operator{z} \subset \omega$,
\item $\dim \operator{(H)} < +\infty$.
\end{enumerate}
И при этом решение задачи \eqref{dif_form} - \eqref{dif_form} экспоненциально стремится к нулю при $t \rightarrow + \infty$
\end{frame}


\begin{frame}
\frametitle{Конструкция оператора управления}

\begin{block}{}
\begin{equation}\label{basis}
	w_j = w_j(x) = \sqrt{2}\sin{\pi j x}, \quad x \in (0, 1), \quad j=1, 2,..
\end{equation}
\end{block}

Рассмотрим следующие операторы проектирования : $P_m : H \rightarrow H_m$, $Q_m : H \rightarrow H_m^{\perp}$.

\begin{block}{}
\begin{equation}
	P_m u = \sum \limits_{j=1}^{m} {(u, w_j) w_j}
\end{equation}

\begin{equation}
	Q_m u = (I - P_m)u(x) = \sum \limits_{j=m + 1}^{\infty} {(u, w_j) w_j}
\end{equation}
\end{block}

\end{frame}

\begin{frame}
\frametitle{Конструкция оператора управления}

В качестве оператора стабилизиции будем рассматривать следующий конечномерный оператор

\begin{block}{}
\begin{equation}
	\operator{z} = -r\chi_{\omega}P_mz, \quad r > 0
\end{equation}
\end{block}
Здесь 
\begin{block}{}
	\begin{equation}
    \begin{matrix}
	    \chi_{\omega}(x) & =
	    & \left\{
	    \begin{matrix}
	    0, & \mbox{если } x \notin \omega, \\
	    1, & \mbox{иначе. }
	    \end{matrix} \right.
	    \end{matrix}
    \end{equation}
\end{block}

В работе доказано, что существуют подходящие параметры $m \in \mathbb{N}$, $r = r_m > 0$, при которых $\operator{}$ обеспечивает стабилизацию неустойчивого решения.

\end{frame}


\begin{frame}
\frametitle{Численные эксперименты}
Предложена следующая неявная разностная схема \\
\begin{block}{}
	\begin{equation}\label{scheme}
		\frac{u^{j + 1}_i - u^j_i}{\tau} - \frac{u_{i + 1}^{j + 1} - 2u_{i + 1}^{j} + u_{i + 1}^{j - 1}}{h^2} - \alpha u^j_i + r\chi_{\omega}P_m u^j_i = 0
	\end{equation}
\end{block}

Оператор $P_m$ аппроксимиран с помощью формулы Филона\\

\end{frame}



\begin{frame}
\frametitle{Численные эксперименты. Пример 1}

\begin{figure}[H]
  \centering
  \includegraphics[width=9cm]{sin_pix}
  \def\figurename{рис}
  \caption{Cистема при $\alpha = \pi^2 - 2$}
\end{figure}

\end{frame}


\begin{frame}
\frametitle{Численные эксперименты. Пример 1}

\begin{figure}[H]
  \centering
  \includegraphics[width=9cm]{sin_un}
  \def\figurename{рис}
  \caption{Неустойчивость при $\alpha = \pi^2 + 0.01$}
\end{figure}



\end{frame}

\begin{frame}
\frametitle{Численные эксперименты. Пример 1}

\begin{block}{}
\begin{equation}
	T_{\tau} = \max\limits_{x \in (0, 1)}{u(x, \tau)}
\end{equation}
\begin{equation}
	m = 1, \, \omega = (0, 0.1) \, \text{и} \, \alpha = \pi^2 + 0.01
\end{equation}
\end{block}

\begin{figure}[H]
  \centering
  \includegraphics[width=3in]{depend}
  \caption{Зависимость $T$ от $r$ при $m = 1$}
\end{figure}


\end{frame}

\begin{frame}
\frametitle{Численные эксперименты. Пример 1}

\begin{figure}[H]
  \centering
  \includegraphics[width=9cm]{sin_st}
  \def\figurename{рис}
  \caption{Стабилизация при подборе параметров $r = 2.5$ и $m = 1$}
\end{figure}


\end{frame}

\begin{frame}
\frametitle{Численные эксперименты. Пример 1}

\begin{figure}[H]
  \centering
  \includegraphics[width=3in]{dependes}
  \def\figurename{рис}
  \caption{Зависимость $T$ от $r$ при разных $m$}
\end{figure}

\end{frame}

\begin{frame}
\frametitle{Численные эксперименты. Пример 2}

\begin{figure}[H]
  \centering
  \includegraphics[width=9cm]{ex2_un}
  \def\figurename{рис}
  \caption{Неустойчивость при $\alpha = \pi^2 + 3$}
\end{figure}

\end{frame}

\begin{frame}
\frametitle{Численные эксперименты. Пример 2}

\begin{figure}[H]
  \centering
  \includegraphics[width=3in]{depend2}
  \caption{Зависимость $T$ от $r$ при $\omega = (0, 0.4)$ и при разных $m$}
\end{figure}

\end{frame}

\begin{frame}
\frametitle{Численные эксперименты. Пример 2}

\begin{figure}[H]
  \centering
  \includegraphics[width=9cm]{ex2_st}
  \def\figurename{рис}
  \caption{Стабилизация при подборе параметров $r = 2.5$ и $m = 1$}
\end{figure}

\end{frame}

\begin{frame}
\frametitle{Численные эксперименты. Пример 3}

\begin{figure}[H]
  \centering
  \includegraphics[width=9cm]{ex3_un}
  \def\figurename{рис}
  \caption{Решение задачи при $\alpha = \pi^2$}
\end{figure}


\end{frame}

\begin{frame}
\frametitle{Численные эксперименты. Пример 3}

\begin{figure}[H]
  \centering
  \includegraphics[width=9cm]{ex3_st}
  \def\figurename{рис}
  \caption{Решение задачи при $\alpha = \pi^2$ при $m = 1$, $r = 1$}
\end{figure}  
\end{frame}

\begin{frame}
\frametitle{Заключение}

При выполнении цели данной курсовой работы были выполнены следующие задачи:
\begin{itemize}
	\item Изучена литература, связанная с описанием и решением задачи стабилизации параболической системы;
	\item Подробно разобран метод стабилизацей локальным конечномерным управлением с обратной связью;
	\item Мною были доказаны 2 небольшие леммы полезные для доказательства стабилизируемости системы;
	\item Численная реализация метода стабилизации, в том числе : написание программы для вычисления разностной схемы, метода Филона.
	\item Проделаны численные эксперименты для, показывающие эффективность данного метода стабилизации
\end{itemize}

\end{frame}

\begin{frame}
	\begin{center}
	\Huge Спасибо за внимание
	\end{center}
\end{frame}

\end{document}
