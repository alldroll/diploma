\subsection{Численная реализация алгоритма}
\vspace{1em}

В настоящем параграфе приведена численная реализация предложенного алгоритма стабилизации.\\

Рассмотрим задачу с стабилизирующим оператором

\begin{equation}\label{sys}
u_t - u_{xx} - \alpha(x) u = -r\chi_{\omega}P_m u, \quad 0 < x < 1, \quad t > 0
\end{equation}
К уравнению \eqref{sys} добавим начальное и граничные условия
\begin{gather}\label{s_control}
u(0, t) = u(1, t) = 0 \\*
u(x, 0) = u_{0}(x) \in H \nonumber
\end{gather}

Для \eqref{sys} запишем неявную разностную схему \\
\begin{equation}\label{scheme}
  \frac{u^{j + 1}_i - u^j_i}{\tau} - \frac{u_{i + 1}^{j + 1} - 2u_{i}^{j + 1} + u_{i - 1}^{j + 1}}{h^2} - \alpha(x_{i}) u_{i}^{j + 1} + r\chi_{\omega}P_m u^j_i = 0
\end{equation}
Запишем аппроксимацию начального и граничных условий:\\
\begin{gather}
u_i^0 = u_0(x_i) \\*
u_1^{j+1} = u_N^{j+1} = 0 \nonumber
\end{gather}
Вспомним, что оператор проектирования имеет вид :\\
$P_m u = \sum \limits_{j=1}^{m} {(u, w_j) w_j} = \sqrt{2} (\sum \limits_{j=1}^{m} {C_k \sin{(\pi k x)}})$. Здесь $C_k = \sqrt{2} \int\limits_0^1{u(s)\sin{(\pi k s)} ds}$\\
Заметим, что $C_k$ - это интеграл от быстро осциллирующей функции вида
\begin{equation}
  \int\limits_a^b{f(x) e^{i\omega x} dx} \approx \int\limits_a^b{L_3(x) e^{i\omega x} dx}
\end{equation}
Поскольку, функция $f$ является гладкой, то на $[a, b]$ она легко приближается с известной погрешностью интерполяционными многочленами. Пусть для определенности, это интерполяционный многочлен в форме Лагранжа :
\begin{equation}
  L_3(x) = P_1(x)f(x_1) + P_2(x)f(x_2) + P_3(x)f(x_3)
\end{equation}
построенный по узлам $x_1 = a$, $x_2 = \frac{a + b}{2}$, $x_3 = b$. $P_i$ - многочлены второй степени, не зависящие от функции $f$. Данный метод приближенного интегрирования называется формулой Филона. Именно этим способом и будем аппроксимировать оператор $P_m$.\\

Для решения неявной схемы \eqref{scheme} воспользуемся методом прогонки.