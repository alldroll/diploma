\section{Численные эксперименты}
\vspace{1em}
В настоящем параграфе приведены численные примеры, демонстрирующие предложенный алгоритм стабилизации\\

Рассмотрим задачу с построенным выше стабилизирующим оператором

\begin{equation}\label{sys}
u_t - u_xx - \alpha u = -r\chi_{\omega}P_m u, \quad 0 < x < 1, \quad t > 0
\end{equation}
К уравнению \eqref{sys} добавим начальное и граничные условия
\begin{gather}\label{s_control}
u(0, t) = u(1, t) = 0 \\*
u(x, 0) = u_{0}(x) \in H \nonumber
\end{gather}

% Для решения задачи стабилизиции, выберем $\omega \in (0, 1)$, $m \in \mathbb{N}$ и подбёрем такой параметр $r > 0$, чтобы решение системы \eqref{sys} стремилось к 0 при $t \rightarrow \infty$\\ 
Для численной реализации, запишем для \eqref{sys} неявную разностную схему :\\
\begin{equation}\label{scheme}
	\frac{u^{j + 1}_i - u^j_i}{\tau} - \frac{u_{i + 1}^{j + 1} - 2u_{i + 1}^{j} + u_{i + 1}^{j - 1}}{h^2} - \alpha u^j_i + r\chi_{\omega}P_m u^j_i = 0
\end{equation}
Запишем аппроксимацию начального и граничных условий:\\
\begin{gather}
u_i^0 = u_0(x_i) \\*
u_1^{j+1} = u_N^{j+1} = 0 \nonumber
\end{gather}
Вспомним, что оператор проектирования имеет вид :\\
$P_m u = \sum \limits_{j=1}^{m} {(u, w_j) w_j} = \sqrt{2} (\sum \limits_{j=1}^{m} {C_k \sin{\pi k x}})$. Здесь $C_k = \sqrt{2} \int\limits_0^1{u(s)\sin{(\pi k s)} ds}$\\
Заметим, что $C_k$ - это интеграл от быстро осциллирующей функции вида
\begin{equation}
	\int\limits_a^b{f(x) e^{i\omega x} dx} \approx \int\limits_a^b{L_3(x) e^{i\omega x} dx}
\end{equation}
Поскольку, функция $f$ является гладкой, то на $[a, b]$ она легко приближается с известной погрешностью интерполяционными многочленами. Пусть для определенности, это интерполяционный многочлен в форме Лагранжа :
\begin{equation}
	L_3(x) = P_1(x)f(x_1) + P_2(x)f(x_2) + P_3(x)f(x_3)
\end{equation}
построенный по узлам $x_1 = a$, $x_2 = \frac{a + b}{2}$, $x_3 = b$. $P_i$ - многочлены второй степени не зависящие от функции $f$. Данный метод приближенного интегрирования, называется формулой Филона. Именно этим способом и будем аппроксимировать оператор $P_m$\\

Для решения неявной схемы \eqref{scheme}, воспользуемся методом прогонки.

\newtheorem{exmp}{Пример}[section]
\begin{exmp}
\end{exmp}
В качестве начального условия $u(x, 0) = \sin{\pi x}$, а граничные данные возмем равные нулю. Вспомним, что при $\alpha < \pi^2$ система устойчива. На рис.1 приведен график решения задачи \eqref{sys} - \eqref{s_control} без управления $m = 0$, $r = 0$ при $\alpha = \pi^2 - 2$

\begin{figure}[hb]
  \centering
  \includegraphics[width=5in]{sin_pix}
  \caption{Cистема при $\alpha = \pi^2 - 2$}
\end{figure}

На рис.2 продемонстрируем неустойчивость нулевого решения при $\alpha = \pi^2 + 0.01$

\begin{figure}[H]
  \centering
  \includegraphics[width=5in]{sin_un}
  \caption{Неустойчивость при $\alpha = \pi^2 + 0.01$}
\end{figure}

Теперь попробуем стабилизировать нашу систему. Фиксируем $\omega = (0, 0.1)$. Вспомним, что необходимо подобрать параметры $m$, $r_m$ так, чтобы $\beta > 0$ и $q > 0$. Рассмотрим подробнее $q = [(\pi(m + 1))^2 - \alpha - 1]$. При заданном $\alpha = \pi^2 + 0.01$, достаточно взять $m = 1$ для выполнения неравенства. Параметр $r$ прийдется подобрать, чтобы решение стремилось к 0.\\

Обозначим за $T_{\tau} = \max\limits_{x \in (0, 1)}{u(x, \tau)}$ и проиллюстрируем (рис. 3) зависимость устойчивости решения от выбора параметра $r$ при фиксированных $m = 1$, $\omega = (0, 0.1)$ и $\alpha = \pi^2 + 0.01$.

\begin{figure}[H]
  \centering
  \includegraphics[width=3in]{depend}
  \caption{Зависимость $T$ от $r$ при $m = 1$}
\end{figure}

\begin{figure}[H]
  \centering
  \includegraphics[width=5in]{sin_st}
  \caption{Стабилизация при подборе параметров $r = 2.5$ и $m = 1$}
\end{figure}

На рис. 4 приведен график решения, который демонстрирует стабилизацию при выборе параметра $r = 2.5$ исходя из рис. 2

Видим, что решение системы \eqref{sys} стремится к $0$ при возрастании $t$\\

Продемонстрируем зависимость устойчивости решения (на рис. 5) от выбора параметра $r$ при разных $m$.

\begin{figure}[H]
  \centering
  \includegraphics[width=4in]{dependes}
  \caption{Зависимость $T$ от $r$ при разных $m$}
\end{figure}

\begin{exmp}
\end{exmp}

Начальные условия $u(x, 0) = x(1 - x)$, а граничные данные возмем равные нулю. Заведомо выберем параметр $\alpha = \pi^2 + 3$ большим. Зафиксируем $\omega = (0, 0.4)$. Необходимо подобрать $m$, таким чтобы $q > 0$. При $m \ge 1$ условие выполняется, поэтому мы фиксируем $m = 1$.

\begin{figure}[H]
  \centering
  \includegraphics[width=7in]{ex2_un}
  \caption{Неустойчивость при $\alpha = \pi^2 + 3$}
\end{figure}

На рис. 6 продемонстрировано, как быстро растет решение задачи \eqref{sys} - \eqref{s_control} при небольшом увеличении $\alpha$.\\

Для стабилизации решения, проиллюстрируем (рис. 7) зависимость устойчивости решения от выбора параметра $r$ для фиксированных $\omega$ и при разных $m$.

\begin{figure}[H]
  \centering
  \includegraphics[width=5in]{depend2}
  \caption{Зависимость $T$ от $r$ при $\omega = (0, 0.4)$ и при разных $m$}
\end{figure}

Из рис. 7, параметр $r$ выбираем равный $2.5$

\begin{figure}[H]
  \centering
  \includegraphics[width=6in]{ex2_st}
  \caption{Стабилизация при подборе параметров $r = 2.5$ и $m = 1$}
\end{figure}

\begin{exmp}
\end{exmp}

Выберем начальные условия $u(x, 0) = \sin({x}) \cos({\frac{\pi x}{2}})$, граничные - нулевыми. Будем рассматривать систему \eqref{sys} - \eqref{s_control} при $\alpha = \pi^2$. Вспомним, что в этом случае, теория говорит о том, что нулевое решение устойчиво по Ляпунову, но неустойчиво ассимптотически.

\begin{figure}[H]
  \centering
  \includegraphics[width=6in]{ex3_un}
  \caption{Решение задачи при $\alpha = \pi^2$}
\end{figure}

На рис. 9 продемонстрировано ассимптотически неустойчивое решение задачи\\

Стабилизируем решение, зафиксировав $\omega = (0, 0.2), $выбрав $m = 1$, $r = 1$ из тех же соображений, что и раньше.

\begin{figure}[H]
  \centering
  \includegraphics[width=6in]{ex3_st}
  \caption{Решение задачи при $\alpha = \pi^2$ при $m = 1$, $r = 1$}
\end{figure}  

На рис. 10 продемонстрировано стабилизация решение начально-краевой задачи \eqref{sys} - \eqref{s_control}