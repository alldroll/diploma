\subsection{Стабилизация конечномерным локальным управлением с обратной связью}

Рассмотрим систему

\begin{equation}\label{ndif_form}
    u_t = u_{xx} + \alpha u, \ 0 < x < 1, \ t > 0
\end{equation}

с начальным и граничными условиями

\begin{gather}
    u(0, t) = u(1, t) = 0, \\*
    u(x, 0) = u_{0}(x) \in H .\nonumber
\end{gather}

Здесь и далее $H = L_2(0, 1)$, $V = H^1_0(0, 1)$.\\
Как показано в \S 1, в случае, когда $\alpha > \pi^2$, нулевое решение уравнения 
\eqref{ndif_form} неустойчиво.
Задача стабилизации параболического уравнения, заданного в ограниченной области,
заключается в построении такого оператора управления, чтобы решение смешанной 
краевой задачи стремилось (при $t \rightarrow \infty$) к заданному стационарному 
решению с предписанной скоростью $\exp (-\delta_0t)$.

\subsection{Конструкция оператора управления}
Сформулируем задачу стабилизации неустойчивого нулевого решения уравнения 
\eqref{ndif_form} за счет локального управления.\\

Пусть $\omega \subset (0, 1)$ - произвольный интеравал такой, что 
$\bar{\omega} \subset (0, 1)$.Задача стабилизации за счет конечномерных локально 
распределённых в $\omega$ управлений заключается в построении оператора 
$\mathcal{R} : H \rightarrow H$ такого, что

\begin{enumerate}
    \item $\forall z \in H \ \supp \operator{z} \subset \omega$,
    \item $\dim \operator{(H)} < +\infty$.
\end{enumerate}

И при этом решение задачи

\begin{gather}
    u_t - u_{xx} - \alpha u = \operator{u} \nonumber\\
    u(0, t) = u(1, t) = 0, \ u(x, 0) = u_0 \nonumber
\end{gather}

экспоненциально стремится к нулю при $t \rightarrow + \infty$\\

\vspace{2em}

Заметим, что функции

\begin{equation}\label{basis}
    w_j = w_j(x) = \sqrt{2}\sin{(\pi j x)}, \ x \in (0, 1), \ j=1, 2, ..
\end{equation}

образуют базис в $H$ и в $V$, причем в $H$ базиc ортонормирован.\\
Через $H_m$ обозначим подпространство в $H$, образованное первыми $m$ функциями 
из \eqref{basis}.\\

Далее рассмотрим следующие операторы проектирования

$$P_m : H \rightarrow H_m, \ Q_m : H \rightarrow H_m^{\perp}$$

\begin{equation}
    P_m u = \sum \limits_{j=1}^{m} {(u, w_j) w_j}
\end{equation}

\begin{equation}
    Q_m u = (I - P_m)u(x) = \sum \limits_{j=m + 1}^{\infty} {(u, w_j) w_j}
\end{equation}

В качестве оператора стабилизиции будем рассматривать следующий конечномерный
оператор

$$\operator{z} = -r\chi_{\omega}P_mz, \ r > 0$$

Здесь
\begin{gather*}
    \begin{matrix}
        \chi_{\omega}(x) & =
        & \left\{
        \begin{matrix}
            0, & \mbox{если } x \notin \omega, \\
            1, & \mbox{иначе. }
        \end{matrix} \right.
    \end{matrix}
\end{gather*}

В следующем параграфе будет доказано, что существуют подходящие параметры 
$m \in \mathbb{N}$, $r = r_m > 0$, при которых $\operator{}$ обеспечивает 
стабилизацию неустойчивого решения.

\subsection{Теоретическое обоснование стабилизации}

Рассмотрим уравнение

\begin{equation}\label{control}
    u_t - u_{xx} - \alpha u = -r\chi_{\omega}\varphi,
\end{equation}

здесь $\varphi = P_mu$. Домножим скалярно обе части \eqref{control} на
$\varphi$. Учтем, что

\begin{gather*}
    (u, \varphi) = \norm{\varphi}^2, \\*
    (u_t, \varphi) = (\varphi_t, \varphi) = \frac{1}{2} \frac{d}{dt}
    \norm{\varphi}^2, \\*
    (u_{xx}, \varphi) = (\varphi_{xx}, \varphi) = -(\varphi_x, \varphi_x) = -
    \norm{\varphi_x}^2, \\*
    (\chi_{\omega}\varphi, \varphi) = \norm{\varphi}^2_{\omega}, 
\end{gather*}

тогда

\begin{equation*}
    \frac{1}{2} \frac{d}{dt} \norm{\varphi}^2  + \norm{\varphi_x}^2 - 
    \alpha \norm{\varphi}^2 + r \norm{\varphi}^2_{\omega} = 0.
\end{equation*}

Воспользуемся неравенством Пуанкаре-Фридрихса-Стеклова

\begin{equation}
    \norm{\varphi}^2 \le \frac{1}{\pi^2} \norm{\varphi_x}^2
\end{equation}

В итоге получаем

\begin{equation}
    \frac{1}{2} \frac{d}{dt} \norm{\varphi}^2 + \pi^2 \norm{\varphi}^2 - 
    \alpha \norm{\varphi}^2 + r \norm{\varphi}^2_{\omega} \le 0
\end{equation}

Приведём полезные для доказательства стабилизируемости леммы.

\newtheorem{lemma}{Лемма}

\begin{lemma}\label{util_lemma}
    Система $\left\{ w_j|_{\omega} \right\}^m_1$ линейно независима.
\end{lemma}

\begin{proof}
    % Пусть $D$ - оператор дифференцирования. Он является линейным отображением(преобразованием) $\mathbb{R}^m$ в $\mathbb{R}^m$. Рассмотрим следующее равенство :\\
    Cистема $\left\{ w_j|_{\omega} \right\}^m_1$ линейно независима, если 
    тождество ввида
    
    \begin{equation}\label{sum_func}
        \sum \limits_{j = 1}^m{c_j w_j(x)} = 0, \ x \in \omega
    \end{equation}

    выполняется только при $c_1 = c_2 = ... = c_m = 0$\\

    Пусть $D$ оператор дифференцирования, действующий в пространстве бесконечно 
    дифференцируемых функций.
    Заметим, что $D^2 w_j = -(\pi k)^2 w_j$. Подействуем  на \eqref{sum_func} 
    оператором $D$ $2l$ раз

    \begin{equation*}
        c_m (\pi m)^{2l} w_m + \sum \limits_{j = 1}^{m - 1}{c_j (\pi j)^{2l} w_j(x)} = 0
    \end{equation*}

    Разделим на $(\pi m)^{2l}$. Тогда

    \begin{equation*}
        c_m w_m + \sum \limits_{j = 1}^{m - 1}{c_j \left(\frac{j}{m}\right)^{2l} w_j(x)}
    \end{equation*}

    Заметим, что при $l \rightarrow +\infty$, правое слагаемое стремится к 0.\\
    Следовательно,

    \begin{equation*}
        c_m w_m = 0
    \end{equation*}

    Функция $w_m(x) = \sin{(\pi m x)}$ не может принимать нулевые значения на 
    целом интервале, следовательно $c_m = 0$. Проделав те же самые рассуждения 
    для $\sum \limits_{j = 1}^{m - 1}{c_j w_j(x)}$, получаем что

    \begin{equation*}
        c_1 = c_2 = ... = c_m = 0.
    \end{equation*}

\end{proof}

\par
\vspace{2ex}

\begin{lemma}\label{main_lemma}
    \begin{equation}
        \gamma = \inf{ \left\{ \norm{z}^2_{\omega} : z = P_mu,\enskip u \in H, 
        \enskip \norm{z} = 1 \ \right\} } > 0,
    \end{equation}

    \par здесь $\norm{z}^2_{\omega} = \int\limits_{\omega}{z^2dx}$.
\end{lemma}

\begin{proof}

    Рассмотрим функцию $f$, определённую на единичной сфере в $\mathbb{R}^m$

    \begin{equation}
       f(c_1, c_2, ..., c_m) = \int \limits_{\omega} {(\sum\limits_1^m {c_jw_j})^2}
       \text{, где } c_1^2 + c_2^2 + ... + c^2_m = 1
    \end{equation}

    По теореме Вейштрасса, существует функция 
    $z_0 = \sum\limits_1^m {c_j^0 w_j} \in H_m$, такая, что

    \begin{equation}
       \gamma = f(c_1^0, c_2^0, ..., c_m^0) = \inf \limits_{c_1^2 + .. + c_m^2 =
       1} {f}.
    \end{equation}

    Заметим, что из линейной независимости 
    $\left\{ \omega_j \right\}^m_1$(по лемме \ref{util_lemma}) следует 
    положительность $\gamma$

\end{proof}

\par
\vspace{2ex}

Выберем число $m \in \mathbb{N}$ так, что

\begin{equation}
    q = [(\pi(m + 1))^2 - \alpha - 1] > 0
\end{equation}

На основании леммы \ref{main_lemma} справедлива следующая оценка

\begin{equation*}
    \frac{1}{2} \frac{d}{dt} \norm{\varphi}^2 + (\pi^2 - \alpha + r\gamma) 
    \norm{\varphi}^2 \le 0.
\end{equation*}

Далее, выберем число $r = r_m > 0$ так, чтобы
$\beta = \pi^2 - \alpha + r\gamma > 0$.
Умножим обе части неравенства на $2e^{2\beta t}$, интегрируем по $t$ и 
получаем следующую важную оценку

\begin{equation}\label{phi_mark}
    \norm{\varphi}^2 \le \norm{\varphi_0}^2 e^{-2\beta t},
\end{equation}

здесь $\varphi_0 = P_m u_0$
\vspace{2em}

Проведём аналогичные выкладки с $\psi = Q_m u$. Домножим обе части 
\eqref{control} скалярно на $\psi$. Учтем, что

\begin{gather*}
    (u, \psi) = \norm{\psi}^2,\\*
    (u_t, \psi) = (\psi_t, \psi) = \frac{1}{2} \frac{d}{dt} \norm{\psi}^2,\\*
    (u_{xx}, \psi) = (\psi_{xx}, \psi) = -(\psi_x, \psi_x) = -
    \norm{\psi_x}^2,\\*
    (\chi_{\omega}\varphi, \psi) = \int_{\omega}{\varphi \psi} \le
    \norm{\varphi} \norm{\psi}
\end{gather*}

С помощью неравенства Пуанкаре-Фридрихса-Стеклова получаем оценку

\begin{equation}
    \norm{\psi}^2 \le (\frac{1}{\pi(m + 1)})^2 \norm{\psi_x}^2,
\end{equation}

тогда

\begin{equation}\label{v2}
    \frac{1}{2} \frac{d}{dt} \norm{\psi}^2 + (\pi(m + 1))^2 \norm{\psi}^2 - 
    \alpha \norm{\psi}^2 \le r \norm{\varphi} \norm{\psi}
\end{equation}

Рассмотрим подробнее \eqref{v2}.\\
Для оценки произведения норм, воспользуемся неравенством Юнга

\begin{equation*}
    \norm{\varphi} \norm{\psi} \le (\frac{\varepsilon \norm{\psi}^2}{2} + 
    \frac{\norm{\varphi}^2}{2 \varepsilon}),
\end{equation*}

здесь $\varepsilon = \frac{2}{r}$. Из \eqref{v2} получаем неравенства

\begin{equation*}
    \frac{1}{2} \frac{d}{dt} \norm{\psi}^2 + (\pi(m + 1))^2 \norm{\psi}^2 - 
    \alpha \norm{\psi}^2 \le r (\frac{\varepsilon \norm{\psi}^2}{2} + 
    \frac{\norm{\varphi}^2}{2 \varepsilon})
\end{equation*}

\begin{equation*}
    \frac{1}{2} \frac{d}{dt} \norm{\psi}^2  + [(\pi(m + 1))^2 - \alpha - 1] 
    \norm{\psi}^2 \le \frac{r^2}{4}\norm{\varphi}^2 \le 
    \frac{r^2}{4}\norm{\varphi_0}^2 e^{-2\beta t}.
\end{equation*}

\begin{equation}
    \frac{1}{2} \frac{d}{dt} \norm{\psi}^2 + q\norm{\psi}^2 \le 
    \frac{r^2}{4}\norm{\varphi_0}^2 e^{-2\beta t}.
\end{equation}

Напомним, что $q > 0$ за счет выбора числа гармоник в операторе $P_m$.
Домножим обе части неравенства на $2e^{2qt}$ и проинтегрируем

\begin{gather*}
    \int\limits_0^t{d(e^{2q\tau}} \norm{\psi}^2) \le 
    \frac{r^2}{2} \norm{\varphi_0}^2 \frac{1}{2q - 2\beta} 
    \int\limits_0^t {e^{(2q -2\beta) \tau} d[(2q -2\beta) \tau]}\\*
    e^{2qt} \norm{\psi}^2 - \norm{\psi_0}^2 \le \frac{r^2}{2}
    \norm{\varphi_0}^2 \frac{1}{2q - 2\beta} e^{(2q -2\beta)t}
\end{gather*}

Следовательно

\begin{equation}\label{psi_mark}
    \norm{\psi}^2 \le \norm{\psi_0}^2 e^{-2qt} + \frac{r^2}{2} 
    \norm{\varphi_0}^2 \frac{1}{2(q - \beta)} e^{-2\beta t}
\end{equation}

Известно, что $u = \varphi + \psi$. Воспользуемся оценками норм $\psi$ и 
$\varphi$ [\eqref{psi_mark}, \eqref{phi_mark}] и получим

\begin{gather*}
    \norm{u}^2 = \norm{\varphi}^2 + \norm{\psi}^2 \le \norm{\varphi_0}^2 
    e^{-2\beta t} + \norm{\psi_0}^2 e^{-2qt} + \frac{r^2}{2} \norm{\varphi_0}^2 
    \frac{1}{2(q - \beta)} e^{-2\beta t}
\end{gather*}

Окончально, оценку стабилизации можно записать в виде

\begin{equation}
    \norm{u}^2 \le \norm{u_0}^2 \left( e^{-2\beta t} + e^{-2qt} + 
    \frac{r^2 e^{-2\beta t}}{4(q - \beta)} \right)
\end{equation}
