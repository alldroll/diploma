\section*{Основные обозначения}
\addcontentsline{toc}{section}{Основные обозначения}
\vspace{1em}

Символы $u_t, u_x, u_{xx}$ - всюду обозначают соотвествующие классические
производные функции $u$\\

$L^{p}(\Omega), \ p \ge 1,$ - банахово пространство (т.е. полное линейное
нормированное пространство), состоящее из всех определенных и измеримых (по
Лебегу) на $\Omega$ функций, имеющих конечную норму
\begin{equation}
    \norm{u}_{p, \Omega} = (\int\limits_{\Omega}{|u|^p dx})^{1/p}
\end{equation}
Норма в $L^2(\Omega)$ обозначим коротко $\norm{.}$, а скалярное произведение
как $(, )$.\\

$W^l_m(\Omega)$(Пространство Соболева) - функциониональное пространство,
состоящее из функций из пространства Лебега, имеющих обобщенные производные 
заданного порядка из $L^m(\Omega)$. В частности
$W^1_2(\Omega)$ состоит из элементов $L^2(\Omega)$, имеющих квадратично
суммируемые по $\Omega$ обобщенные производные первого порядка. В дальнейшим,
через $H^l(\Omega)$ обозначим $W^l_2(\Omega)$.


