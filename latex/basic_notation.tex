%\section*{Основные обозначения}
%\addcontentsline{toc}{section}{Основные обозначения}
%\vspace{3em}

\Abbreviations

\begin{description}

    \item{Символы $u_t, u_x, u_{xx}$ обозначают соотвествующие классические
        производные функции $u$;}

    \item{Через $E'$ обозначим пространство сопряженное к пространству $E$;}

    \item{$L^{p}(\Omega), \ p \ge 1,$ - банахово пространство (т.е. полное линейное
нормированное пространство), состоящее из всех определенных и измеримых (по
Лебегу) на $\Omega$ функций, имеющих конечную норму
\begin{equation*}
    \norm{u}_{p, \Omega} = (\int\limits_{\Omega}{|u|^p dx})^{1/p}
\end{equation*}
Норму в $L^2(\Omega)$ обозначим коротко $\norm{\ .\ }$, а скалярное произведение
как $(\cdot,\cdot)$. Этим же символом будем обозначать отношения двойственности
между $E$ и $E'$;}

    \item{$W^l_m(\Omega)$(Пространство Соболева) - функциониональное пространство,
состоящее из функций из пространства Лебега, имеющих обобщенные производные 
заданного порядка из $L^m(\Omega)$. В частности
$W^1_2(\Omega)$ состоит из элементов $L^2(\Omega)$, имеющих квадратично
суммируемые по $\Omega$ обобщенные производные первого порядка. В дальнейшем
через $H^l(\Omega)$ обозначим $W^l_2(\Omega)$;}

    \item{Здесь и далее, $\Omega = (0, 1) \subset \mathbb{R}$, $H = L^2(\Omega)$ и 
$V = H^1_0(\Omega)$.}

\end{description}


