\section{Введение}

Пусть

\begin{equation}
    \Omega = (0, 1), \quad \omega \subset \Omega
\end{equation}

Неустойчивое уравнение Бюргерса

\begin{gather}
    y_t - \nu y_{xx} + yy_x = f \\*
    y|_{\partial \Omega} = y_{bi} \\*
    y|_{t=0} = y_0
\end{gather}

Стационарное уравнение Бюргерса

\begin{gather}
    y_{sxx} + y_s y_{sx} = f_s \\*
    y|_{s \partial \Omega} = y_{bi}
\end{gather}

\subsection{Задача стабилизации}
Найти оператор управления с обратной связью $\Lambda$ такой, что решение задачи
(1), где $f = f_s + \Lambda (y - y_s)$ экспоненциально сходится к $y_s$

Обозначим за $\varphi = y - y_s$, тогда

\begin{gather}
    \varphi_t - \nu \varphi_{xx} + y_s \varphi_x + \varphi y_{sx} + \varphi
    \varphi_x = \Lambda(\varphi) \\*
    \varphi_0 = y_0 - y_s \\*
    \varphi|_{\partial \Omega} = 0
\end{gather}

\subsection{Формализация}
Здесь и далее
\begin{equation}
    H = L^2(\Omega), \quad V = H^1_0(\Omega), \quad \norm{z}^2 = \int_{\Omega}{z^2dx}
\end{equation}

\begin{equation}
    V \subset H = H^{'} \subset V^{'} = H^{-1}(\Omega), (y, z) =
    \int_{\Omega}{yzdx}
\end{equation}

\begin{equation}
    A: V \rightarrow V^{'}, (Ay, z) = (y_x, z_x)
\end{equation}

\begin{equation}
    B: V \times V \rightarrow V^{'}, \quad (B(y, z), w) = (yz_x, w)
\end{equation}

\begin{equation}
    A w_j = \lambda_j w_j, \quad j = 1, 2.. \quad \lambda_j \rightarrow +\infty
\end{equation}

\begin{equation}
    \lambda_j = (\pi j)^2, \quad w_j(x) = \sqrt{2}\sin{(\pi j x)}
\end{equation}

Заметим что базис ортонормирован в $H$

\begin{equation}
    (w_j, w_k) = \delta_{jk}
\end{equation}

Введем следующие обозначения

\begin{equation}
    H_m = span{w_1, ..., w_m}, \quad P = P_m: H \rightarrow H_m, \quad Q=Q_m = I
    - P
\end{equation}

\subsubsection{Неравенства}

\begin{equation}
    \lambda_1 \norm{v}^2 \le \norm{v_x}^2, \quad \norm{v}^4_{L^4} \le \norm{v}^2
    \norm{v_x}^2
\end{equation}

\begin{equation}
    |(B(y, z), w)| \le \norm{y}_{L^4} \norm{z_x} \norm{w}_{L^4} \le
    \norm{y}^{1/2} \norm{y_x}^{1/2} \norm{z_x} \norm{w}^{1/2} \norm{w_x}^{1/2}
\end{equation}

\begin{equation}
    \norm{w}^2 \le \lambda^{-1}_{m + 1} \norm{w_x}^2 \forall w \in V \cap
    H_m^{\bot}
\end{equation}

\begin{equation}
    |v(x)| \le \norm{v_x}, \quad |(B(y, z), w)| \le \norm{y_x} \norm{z_x}
    \norm{w} \text{или} \norm{y} \norm{z_x} \norm{w_x}
\end{equation}

Оператор локально распределенного управления $\Lambda: H \rightarrow H$, \\
\begin{equation}
    \Lambda(\varphi) = -r \xi_{\omega}P\varphi
\end{equation}


Здесь \\
\begin{gather*}
    \begin{matrix}
        r & =
        & \left\{
        \begin{matrix}
            r_0, & 2 \norm{Q\varphi} \le \norm{P\varphi}, \quad r_0 > 0, \\
            0, & \mbox{иначе. }
        \end{matrix} \right.
    \end{matrix}, \quad
    \begin{matrix}
        \chi_{\omega}(x) & =
        & \left\{
        \begin{matrix}
            0, & \mbox{если } x \notin \omega, \\
            1, & \mbox{иначе. }
        \end{matrix} \right.
    \end{matrix}
\end{gather*}

Перепишем задачу в следующем виде

\begin{gather}
    \varphi' + \nu A\varphi + B(\varphi, y_s) + B(y_s + \varphi, \varphi) =
    \Gamma(\varphi), \quad t > 0 \\*
    \varphi(0) = \varphi_0 = y_0 - y_s
\end{gather}

\subsection{Разрешимость задачи на конечном интервале времени}

\newtheorem{theorem}{Теорема}

\begin{theorem}
    Пусть $y_s \in H^1, \quad \varphi_0 \in H, \quad T > 0$. Тогда $\exists$ решение
    задачи $\varphi \in C([0, T], H) \cap L^2(0, T, V), \varphi' \in L^2(0, T, V')$
\end{theorem}

\begin{remark}
    Пусть $h(\lambda)$ единичная функция Хевисайда, \\
    \begin{gather*}
        \begin{matrix}
            h(\lambda) & =
            & \left\{
            \begin{matrix}
                1, & \lambda \ge 0 \\
                0, & \lambda < 0
            \end{matrix} \right.
        \end{matrix}
    \end{gather*}

    , тогда оператор управления примит следующий вид

    \begin{equation}
        \Lambda(\varphi) = -r_0 h(-2\norm{Q\varphi} +
        \norm{P\varphi})\chi_{\omega}P\varphi
    \end{equation}

\end{remark}


\begin{proof}
    Пусть $\Omega > 0$. Выберим параметры $\sigma$, $r_0$, $m$ такие, что
    $\norm{g(t)}$ ограничена на $[0, +\infty)$. Здесь $g(t) = \varphi(t)e^{\sigma t}$
    \begin{gather*}
        g' + \nu Ag - \sigma g + B(g, y_s) + B(y_s + ge^{-\sigma t}, g) = -r_0 +
        h(\norm{Pg} - 2 \norm{Qg}) + \chi_{\omega}Pg\\*
        g(0) = \varphi_0
    \end{gather*}
    Пусть $p = Pg$, $q = Qg$. Рассмотрим интервал $(t_0, t_1)$, где $\norm{p(t)}
    \le 2 \norm{q(t)}$:
    \begin{gather*}
        g' + \nu Ag - \sigma g + B(g, y_s) + B(y_s + ge^{-\sigma t}, g) = 0, \quad
        t_0 < t < t_1\\
        \frac{1}{2}\frac{d}{dt}\norm{g}^2 + \norm{g_x}^2 - \sigma \norm{g}^2 +
        \frac{1}{2}(y_{sx}, g^2) = 0
    \end{gather*}

    \begin{gather*}
        \norm{g}^2 = \norm{p}^2 + \norm{q}^2 < 5\norm{q}^2 < 5\lambda^{-1}_{m +
        1}\norm{q_x} \le 5 \lambda^{-1}_{m + 1}\norm{g_x}^2
    \end{gather*}

    Пусть 
    \begin{equation}
        C_1 = \sigma + \frac{1}{2}max|y_{sx}|, \quad 5C_1\lambda^{-1}_{m + 1} <
        \frac{\nu}{2}
    \end{equation}

    Тогда отсюда 
    \begin{equation}
        \frac{d}{dt}\norm{g}^2 + \nu \norm{g_x} < 0
    \end{equation}

    Тогда следует

    \begin{equation}
        \norm{g(t_1)} < \norm{g(t_0)}
    \end{equation}

    Далее 
    \begin{gather*}
        \frac{1}{2}\frac{d}{dt}\norm{q}^2 + \nu \norm{q_x}^2 - \sigma \norm{q}^2 +
        ((y_s g)_x, q) + \frac{1}{2}e^{-\sigma t}((g^2)_x, q) = 0\\
        \norm{g_x}^2 = \norm{p_x}^2 + \norm{q_x}^2 \le \lambda_{m} \norm{p}^2 +
        \norm{q_x}^2 \le 5\norm{q_x}^2\\
        |(y_s g)_x, q)| \le C_0 \norm{g} \norm{q_x} < 5 C_0 \lambda^{-1}_{m + 1}
        \norm{q_x}^2, \quad C_0 = max|y_s|\\
        |((g^2)_x, q)| = |(g^2, q_x)| \le \norm{g} \norm{g_x} \norm{q_x}
    \end{gather*}
    
    Пусть теперь $\norm{p(t)} \ge 2 \norm{q(t)}, \quad t \in (t_0, t_1)$\\
    \begin{gather*}
        g' + \nu Ag - \sigma g + (y_s g)_x + \frac{1}{2} e^{-\sigma t} (g^2)_x +
        r_0 \chi_{\omega} P = 0
    \end{gather*}

    Заметим что
    \begin{gather*}
        \gamma |(p, q)_{L^2(\omega)}| \le \norm{p} \norm{q} \le \frac{1}{2}
        \norm{p}^2 \text{следует, что} \\
        \norm{p}^2_{L^2(\omega)} + \gamma (p,
        q)_{L^2(\omega)} \ge \frac{\gamma}{2}\norm{p}^2
    \end{gather*}
    
    \begin{gather*}
        \frac{1}{2} \frac{d}{dt} [ \norm{p}^2 + \nu [ \norm{p_x}^2 - \sigma [
        \norm{p}^2 + \gamma \norm{q}^2] ]  + ( (y_s g)_x, p + \gamma q) + \gamma
    \norm{q} \\* + \gamma\norm{q_x}^2] + \frac{1}{2} e^{-\sigma t} ( (g^2)_x, p +
    \sigma q) + \frac{r_0 \gamma}{2} \norm{p}^2 \le 0\\
    |( (y_s g)_x, p)| \le C_0 \norm{p_x} \norm{g} \le \frac{3}{2} C_0 \norm{p}
    \norm{p_x} \le \frac{\nu}{2} \norm{p_x}^2 + \frac{1}{2\nu}(\frac{3}{2}C_0)^2
    \norm{p}^2
    \end{gather*}

    \begin{equation}
        \norm{g} \le \norm{p} + \norm{q} \le \frac{3}{2} \norm{p}
    \end{equation}

    \begin{gather*}
        |((y_s g)_x, q)| \le C_0 \norm{q_x} \norm{g} \le \frac{3}{2} C_0
        \norm{q_x} \norm{p} \le \frac{\nu}{2} \norm{q_x}^2 + \frac{9C_0^2}{8\nu}
        \norm{p}^2
    \end{gather*}

    \begin{align*}
        \frac{d}{dt}(\norm{p}^2 + \gamma \norm{g}^2) + \nu(\norm{p_x}^2 + \gamma
        \norm{q_x}^2) + (r_0\gamma - (2 + \gamma)\sigma - (1 +
        \gamma)\frac{9C_0^2}{4\nu})\norm{p}^2 \\* \le |(g^2, p_x + \gamma q_x)|
    \end{align*}

    \begin{equation}
        r_0 > C_2 = (1 + \frac{2}{\gamma})\sigma + (1 +
        \frac{1}{\gamma})\frac{9C_0^2}{4\nu}
    \end{equation}

    Оценим правую часть 

    \begin{gather*}
        |(g^2, p_x + \gamma q_x)| = (1 - \gamma)|(g^2, p_x)| = (1 - \gamma)|(p^2
        + 2pq + q^2, p_x)| \\*
        = (1 - \gamma)|(q^2, p_x) - (p^2, q_x)| \le \norm{p_x} \norm{q_x}
        (\norm{p} + \norm{q}) \\* \le
        \frac{1}{2\sqrt{\gamma}}(\norm{p_x}^2 + \gamma \norm{q_x}^2)
        \frac{2}{\sqrt{\gamma}}\sqrt{\norm{p}^2 + \gamma \norm{q}^2}
    \end{gather*}
    т.к. $\norm{p} + \norm{q} \le \frac{2}{\sqrt{\gamma}}\sqrt{\norm{p}^2 +
        \gamma \norm{q}^2}$ отсюда следует, что

    \begin{equation}
        \frac{d}{dt}(\norm{p}^2 + \gamma \gamma{q}^2) + (\norm{p_x}^2 + \gamma
        \norm{q_x}^2) (\nu - \frac{1}{\nu} \sqrt{\norm{p}^2 + \gamma \norm{q}^2}
        \le 0
    \end{equation}

    
\end{proof}
