\subsection{Конструкция управления}
Для сведения начально-краевой задачи \eqref{fluct} к задаче Коши для уравнения
с операторными коэффициентами определим отображения 

\begin{equation}
    A: V \rightarrow V^{'}, \ (Ay, z) = (y_x, z_x)
\end{equation}

\begin{equation}
    B: V \times V \rightarrow V^{'}, \ (B(y, z), w) = (yz_x, w)
\end{equation}

которые выполняются для всех $y, z, w$ из пространства $V$. Для функций из
пространства $V$ справедливы следующие неравенства, которые будут далее
использоваться

\begin{gather}
    \lambda_1 \norm{v}^2 \le \norm{v_x}^2, \ \norm{v}^4_{L^4} \le \norm{v}^2
    \norm{v_x}^2\\
    |(B(y, z), w)| \le \norm{y}_{L^4} \norm{z_x} \norm{w}_{L^4} \le
    \norm{y}^{1/2} \norm{y_x}^{1/2} \norm{z_x} \norm{w}^{1/2}
    \norm{w_x}^{1/2}\\
    \norm{w}^2 \le \lambda^{-1}_{m + 1} \norm{w_x}^2, \ \forall w \in V \cap
    H_m^{\bot}\\
    |v(x)| \le \norm{v_x}, \ |(B(y, z), w)| \le \norm{y_x} \norm{z_x}
    \norm{w} \text{или} \norm{y} \norm{z_x} \norm{w_x}
\end{gather}

В дальнейшем, если $X$ - банахово пространство, то через $L^p(0, T;X)$(соотв.
$C([0, T];X))$ обозначается пространство $L^p$ (соотв. непрерывных) функций,
определенных на $[0, T]$ со значениями в $X$.

Оператор стабилизирующего управления имеет следующую структуру

\begin{equation}
    \Lambda(\varphi) = -r \chi_{\omega}h(\mu)P\varphi
\end{equation}


Здесь \\
\begin{gather*}
    \begin{matrix}
        h(\mu) & =
        & \left\{
        \begin{matrix}
            0, & \mu < 0, \\
            1, & \mu > 0, \\
            [0, 1], & \mu = 0
        \end{matrix} \right.
    \end{matrix} \quad
    \begin{matrix}
        \chi_{\omega}(x) & =
        & \left\{
        \begin{matrix}
            0, & \mbox{если } x \notin \omega, \\
            1, & \mbox{иначе. }
        \end{matrix} \right.
    \end{matrix}
\end{gather*}

$r > 0$, $\chi_{\omega}$ - характеристическая функция области $\omega$,
$h(\mu)$ - многозначная функция Хевисайда\\

Используя введенные операторы, система \eqref{fluct} стандартным образом
переписывается $(\varphi' = \frac{d\varphi}{dt})$

\begin{gather}\label{operator_fluct}
    \varphi' + \nu A\varphi + B(\varphi, U) + B(U + \varphi, \varphi) =
    \Lambda(\varphi), \ t > 0 \\*
    \varphi|_{t = 0} = \varphi_0 = u_0 - U
\end{gather}

\subsection{Разрешимость задачи на конечном интервале времени}

\newtheorem{theorem}{Теорема}

\begin{theorem}
    Пусть $U \in W^1_{\infty}, \ \varphi_0 \in H, \ T > 0$. 
    Тогда $\exists$ решение задачи \eqref{operator_fluct} такое, что
    $$\varphi \in C([0, T], H) \cap L^2(0, T, V), \ \varphi' \in L^2(0, T, V')$$
\end{theorem}

%\begin{proof}
%\end{proof}

\subsection{Конечномерная стабилизация}

Выясним условия, выполнение которых гарантирует, что нулевое состояние
$\varphi_s = 0$ является экспоненциально устойчивой особой точкой динамической
системы \eqref{operator_fluct}\\

На основании леммы 2, справедливо следующая оценка

\begin{equation}\label{lemma2_mark}
    \norm{Pv}^2_{L^2(\omega)} \le \gamma \norm{Pv}^2, \ \forall v \in H
\end{equation}

Пусть $\Omega > 0$. Выберим параметры $\sigma$, $r_0$, $m$ такие, что
$\norm{g(t)}$ ограничена на $[0, +\infty)$. Здесь $g(t) = \varphi(t)e^{\sigma t}$, 
$\varphi$ - решение задачи \eqref{operator_fluct}. Функция $g$ удовлетворяет 
условиям

\begin{gather}\label{g_fluct}
    g' + \nu Ag - \sigma g + B(g, U) + B(U + ge^{-\sigma t}, g) = \Lambda(g)\\*
    g(0) = \varphi_0
\end{gather}

Обозначим $p = Pg$, $q = Qg$. Пусть на некотором интервале $(t_0, t_1)$
справедливо неравенство $\norm{p(t)} \le 2 \norm{q(t)}$, т.е. $\Lambda(g) = 0$.
Умножим скалярно первое уравнение в \eqref{g_fluct} на $g$ и заметим, что $(B(U
+ g e^{-\sigma t}, g), g) = -(U_x, g^2) / 2$. Тогда

\begin{gather*}
    \frac{1}{2}\frac{d}{dt}\norm{g}^2 + \norm{g_x}^2 = \sigma \norm{g}^2 -
    \frac{1}{2}(U_x, g^2) \le C_1\norm{g}^2
\end{gather*}

Здесь $C_1 = \sigma + \frac{1}{2}\norm{U_{x}}_{L^2(\Sigma)}$. На рассматриваемом 
интервале справделиво следующая оценка

\begin{gather*}
    \norm{g}^2 = \norm{p}^2 + \norm{q}^2 < 5\norm{q}^2 < 5\lambda^{-1}_{m +
    1}\norm{q_x} \le 5 \lambda^{-1}_{m + 1}\norm{g_x}^2
\end{gather*}

Следовательно, если 

\begin{equation}
    10C_1 \lambda_{m + 1}^{-1} < \nu
\end{equation}

то справедливо неравенство

\begin{equation}
    \frac{d}{dt}\norm{g}^2 + \nu \norm{g_x} < 0
\end{equation}

Тогда следует

\begin{equation}\label{8}
    \norm{g(t_1)} < \norm{g(t_0)}
\end{equation}

Поэтому $\norm{g(t_1)} < \norm{g(t_0)}$\\

%Далее 
%\begin{gather*}
    %\frac{1}{2}\frac{d}{dt}\norm{q}^2 + \nu \norm{q_x}^2 - \sigma \norm{q}^2 +
    %((U g)_x, q) + \frac{1}{2}e^{-\sigma t}((g^2)_x, q) = 0\\
    %\norm{g_x}^2 = \norm{p_x}^2 + \norm{q_x}^2 \le \lambda_{m} \norm{p}^2 +
    %\norm{q_x}^2 \le 5\norm{q_x}^2\\
    %|(U g)_x, q)| \le C_0 \norm{g} \norm{q_x} < 5 C_0 \lambda^{-1}_{m + 1}
    %\norm{q_x}^2, \ C_0 = max|U|\\
    %|((g^2)_x, q)| = |(g^2, q_x)| \le \norm{g} \norm{g_x} \norm{q_x}
%\end{gather*}

Пусть теперь $\norm{p(t)} \ge 2 \norm{q(t)}, \ t \in (t_0, t_1)$, т.е.
стабилизирующее управление включено. В этом случае, учитывая
\eqref{lemma2_mark}, получаем

\begin{gather*}
    \gamma |(p, q)_{L^2(\omega)}| \le \norm{p} \norm{q} \le \frac{1}{2}
    \norm{p}^2\\
    (p, p + \gamma q)_{L^2(\omega)} = \norm{p}^2_{L^2(\omega)} + \gamma(p,
    q)_{L^2(\omega)} \ge \frac{\gamma}{2}\norm{p}^2.
\end{gather*}

Умножим скалярно первое уравнение в \eqref{g_fluct} на $p + \gamma q$ и
воспользуемся последним неравенством. Тогда

\begin{gather*}
    %\frac{1}{2} \frac{d}{dt} [ \norm{p}^2 + \nu [ \norm{p_x}^2 - \sigma [
    %\norm{p}^2 + \gamma \norm{q}^2] ]  + ( (U g)_x, p + \gamma q) + \gamma
%\norm{q} \\* + \gamma\norm{q_x}^2] + \frac{1}{2} e^{-\sigma t} ( (g^2)_x, p +
%\sigma q) + \frac{r_0 \gamma}{2} \norm{p}^2 \le 0\\
%|( (U g)_x, p)| \le C_0 \norm{p_x} \norm{g} \le \frac{3}{2} C_0 \norm{p}
%\norm{p_x} \le \frac{\nu}{2} \norm{p_x}^2 + \frac{1}{2\nu}(\frac{3}{2}C_0)^2
%\norm{p}^2
    \frac{1}{2} \frac{d}{dt} (\norm{p}^2 + \gamma \norm{q}^2) + \nu
    (\norm{p_x}^2 + \gamma \norm{q_x}^2) - \sigma (\norm{p}^2 + \gamma
    \norm{q}^2) + \frac{r\gamma}{2} \norm{p}^2 \le \\
    (Ug, p_x + \gamma q_x) +
    \frac{1}{2} e^{-\sigma t}(g^2, p_x + \gamma q_x)
\end{gather*}

Для оценки правой части используем неравенства: $\norm{g} \le \norm{p} + 
\norm{q} \le \frac{3}{2} \norm{p}$

\begin{gather*}
    |((U g)_x, p_x + \gamma q_x)| \le \frac{3}{2} C_0 \norm{p}(\norm{p_x} +
    \gamma \norm{q_x}) \le \\
    \frac{\nu}{2} (\norm{p_x}^2 + \gamma \norm{q_x}^2) +
    \frac{9C_0^2}{8\nu}(1 + \gamma)\norm{p}^2\\
    (g^2, p_x + \gamma q_x) = (1 - \gamma)(g^2, p_x) = (1 - \gamma)((q^2, p_x) -
    (p^2, q_x)) \le \\
    \norm{p_x}\norm{q_x}(\norm{p} + \norm{q}) \le \frac{1}{\gamma}(\norm{p_x}^2
    + \gamma \norm{q_x}^2)\sqrt{\norm{p}^2 + \gamma \norm{q}^2}
\end{gather*}

Следовательно

\begin{gather*}
    \frac{d}{dt}(\norm{p}^2 + \gamma \norm{q}^2) + (\norm{p_x}^2 + \gamma
    \norm{q_x}^2)(\nu - \frac{1}{\nu}\sqrt{\norm{p}^2 + \gamma \norm{q}^2} + \\
    (r \gamma - C_2)\norm{p}^2 \le 0
\end{gather*}

Здесь $C_2 = (2 + \gamma)\sigma - \frac{9C_0^2}{4\nu}(1 + \gamma)$. Если
параметр $r$ выбрать так, что 

\begin{equation}
    r\gamma \ge C_2
\end{equation}

то справедливо неравенство

\begin{equation}
    \frac{d}{dt}(\norm{p}^2 + \gamma{q}^2) + (\norm{p_x}^2 + \norm{q_x}^2)(\nu -
    \frac{1}{\nu}\sqrt{\norm{p}^2+ \gamma \norm{q}^2} \le 0
\end{equation}

Поэтому $(\norm{p}^2 + \gamma \norm{q}^2)|_{t=t_1} < (\gamma \nu)^2$, если
$(\norm{p}^2 + \gamma \norm{q}^2)|_{t=t_0} < (\gamma \nu)^2$
