\section{Обзор}
\vspace{1em}

% \subsection{Стационарное решение}

% \begin{equation}
%   u_t = u_{xx} - uu_x  
% \end{equation}

% Пусть u не зависит от $t$, тогда 

% \begin{equation}
%   u_{xx} = uu_x
% \end{equation}

% Обозначим $u_x = p$, $u_{xx} = pp_u$, получаем, $u = p_u$ и тогда

% \begin{equation}
%   2u_x = u^2 + C_0
% \end{equation}

% Предлагаем $u_x \neq 0$ и $C_0 = -a^2 < 0$, тогда $dx = \frac{du}{u^2 - a^2}$

% \begin{equation}
%   ax = \ln{C_1 |\frac{a - u}{a + u}|}
% \end{equation}

% где $C_1 = |\frac{a + u(0)}{a - u(0)}|$\\
% Если в добавок, $|u| < a$, т.е.($u_x < 0$), тогда

% \begin{equation}
%   u = a \frac{C_1 - e^{ax}}{C_1 + e^{ax}} = -a \tanh{\frac{a}{2}(x - \frac{2}{a} \arctanh{\frac{u(0)}{a}})}   
% \end{equation} 

% Если  $|u| > a$, т.е.($0 < 2u_x < u^2$), тогда

% \begin{equation}
%   u = a \frac{C_1 + e^{ax}}{C_1 - e^{ax}} = -a \coth{\frac{a}{2}(x - \frac{2}{a} \arccoth{\frac{u(0)}{a}})}   
% \end{equation} 

% Теперь пусть $C_0 = a^2 > 0$, тогда $dx = \frac{du}{u^2 + a^2}$

% \begin{equation}
%   \frac{ax}{2} = \arctan{\frac{u}{a}} + C_1
% \end{equation}

% , где $C_1 = -\arctan{\frac{u(0)}{a}}$, отсюда 

% \begin{equation}
%   u = a \tan{\frac{ax}{2} + C_1} = -a \cot{\frac{a}{2}(x - \frac{2}{a}\arccot{\frac{u(0)}{a}})}
% \end{equation}

% Теперь $C_0 = 0$, если $u_x = 0$, тогда $u = const$. 

\subsection{Нелинейное управление уравнения Бюргерса}

Обзор - Мы рассматриваем проблему стабилизации неустойчивых ``shock-like" равновесных profile вяского Уравнения Бюргерса с приведением в действие (actuation) на границах. Эти равенства использующие стандартные radiation граничные условия неустойчивы (даже локально). Используя нелинейные пространственно-масштабируемые трансформации (которые состоят из трех ингредиентов, один из которых нелинейная интегральная трансформация Hopf-Cole) и линейная конструкция backstepping, мы разработали явное нелинейное full-state(полный состояние) управление которое достигает экспонентциальную устойчивость, с областью притяжения, для которых мы дали оценку. Область притяжения (region of attraction) не является сплошным (целое, полное) пространством, т.к. уравнение Бюргерса, как известно, не может быть глобально управляемым, однако, результат устойчивости достигается сильнее, чем бесконечно-малое локально. В сопроводительном документе мы рассмотрим стабилизацию с обратной связью, для которых мы разрабатываем нелинейного наблюдателя с границей зондирования, и решаем проблемы траектории поколения и отслеживания.

\subsubsection{Введение}

Мы изучаем проблему нелинейного управления для вязкого уравнения Бюргерса, которое рассматривается как основная модель нелинейного конвективного явления, такие как те, которые возникают в уравнениях Навье-Стокса. В то время как модель Бюргерса не в состоянии захватить всю сложность турбулентности, его нелинейность делает его сложным и отправной точкой на развитие проектов для более реалистичных уравнений Навье-Стокса. Мы рассмотрим семейство стационарных решений называемые shock profiles, которые неустойчивы и не стабилизируется (даже локально) простыми способами такими как граничное radiation condition (условия излучения). Мы достигаем экспонентциальную устойчивость (в норме $L^2$) shock profiles, используя 2 входных (исходных) управления  (одно из которых граничное).

Рание результаты с линейным управлением уравнения Бюргерса были представлены в работах [4], достигнута локальная устойчивость. Оптимальное управление было рассмотрена в работах [7]. В [6] linear static collocated output feedback (так называемые граничные условия излучения [radiation]) доказано что достигается локальная $L^2$ экспонентциальная устойчивость. В [20] этот результат расширяется на $L^{\infty}$. Глобальная устойчивость представлена в [13], используются нелинейные граничные условия, и расширенные на KdV, Kuramoto-Sivashinsky, адаптивное управление, и другие проблемы в [1], [2], [15], [16], [17], [18]. Ссылки на [10], [11] иследуют нулевую управляемость уравнения Бюргерса с одной и двумя входами соответственно, получение оценки на минимальное время стабилизации. В работе [3] проблема стабилизации решается с помощью нелинейных методов снижения модель, с в-области срабатывания.

Наш недавний проект для нелинейных параболических уравнений в частных производных [22], [23] (без конвективных нелинейности) основан на обратной линеаризующие нелинейном преобразовании в виде рядов Вольтерра и мы открыли аллею для полностью нелинейных конструкций для уравнений в частных производных. В этой статье мы следуем концептуально аналогичную стратегию (хотя довольно сильно отличается в его исполнении) и находим нелинейное пространственно-масштабируется преобразование (на основе трех составляющих, одной из которых является Коула-Хопфа нелинейное преобразование [8]), что превращает систему (с помощью одного из двух краевых элементов управления) в линейной реакционно-диффузионной PDE. Это PDE стабилизируется с помощью линейного подхода backstepping [19], что дает право управления, которое является нелинейным в исходном состоянии переменной. Мы предоставляем смету области притяжения для системы с обратной связью. Эта оценка конечна, так как система Бюргерса сожалению не является глобально управляемой [10], [11]. Проиллюстрируем наши теоретические результаты с числовыми примерами. В первым параграфе показано, с помощью исследования собственных чисел, так и через моделирования нелинейной системы, что radiation feedback не стабилизируется, даже локально, для достаточно больших shock profiles. Стабилизационные свойства законам обратной связи показаны на моделировании. 

В другой статье [14] мы проводим стабилизации обратной выхода (output feedback stabilization), для которых мы используем нелинейного наблюдателя, который использует инъекцию выходного ошибки оценки по одной из границ. Мы также решить проблему траектории поколения и отслеживания для уравнения Бюргерса.

\subsubsection{Уравнение Бюргерса}

Рассмотрим вязкое уравнение Бюргерса

\begin{equation}
  u_t = u_{xx} - u_xu
\end{equation}

где $x \in [0, 1]$, $u_x(0, t) = \omega_0(t)$,\quad $u_x(1, t) = \omega_1(t)$

здесь $\omega_0$ и $\omega_1$ входные управления. Дальше мы отбросим $(x, t)$ в контексте.

Нас интересует семейство стационарных решений shock-like

\begin{equation}\label{static}
  U(x) = -2\sigma\tanh{(\sigma(x - \frac{1}{2}))}
\end{equation}

, где параметр $\sigma \ge 0$

% TODO include fig1

Замечание: Мы указали что все результаты стабилизации и конструкция наблюдателя в этой статье могут быть получены для любых других непрерывных равновесных профилей $U(x)$. Однако, мы концентрируемся на симметричных профилях таких как самый интересный класс равновесия, где open-loop неустойчивость является выраженной. Мы также показываем, что уравнение Бюргерса обычно изучается в форме $u_t = \epsilon u_{xx} - u_xu$, в этой статье мы рассматриваем только проблему с $\epsilon = 1$. Все результаты в этой статье могут быть получены для любого $\epsilon \ne 1$ и зависить от важного (необходимого) $\epsilon$. Профиль $U(x) = -2\epsilon\sigma\tanh{(\sigma(x - \frac{1}{2}))}$, результаты котрого в open-loop собственных значений просто scales (маштабируются?) от этого параметра. Основной эффект $\epsilon \ne 1$ результатов нелинейных аспектов влияющий на актуальный размер области сжатия, но не фундаментальную форму размеров оценки.

Продолжим, сначало мы заметим, что 

\begin{equation}
  U'(x) = -2\sigma^2(1 - \tanh^2{(\sigma(x - \frac{1}{2}))})
\end{equation}

Отсюда

\begin{equation}
  U'(0) = U'(1) = -2\sigma^2(1 - \tanh^2{(\frac{\sigma}{2})})
\end{equation}

Тогда из граничных условий, мы получаем что 

\begin{equation}
  \omega_0 = \omega_1 = -2\sigma^2(1 - \tanh^2{(\frac{\sigma}{2})}) \le 0
\end{equation}

что они являются константами в open-loop управления $\omega_0(t)$ и $\omega_1(t)$.

Позвольте нам обозначить флуктуацию переменых вокруг shock profile такую как $\hat{u}(x, t) = u(x, t) - U(x)$, где $U(x)$ определена в \eqref{static}. Тогда уравнение Бюргерса в новой переменной $\hat{u}$ перепишется следующим образом

\begin{equation}\label{fluct}
  \hat{u}_t = \hat{u}_{xx} - U(x)\hat{u}_x - U'(x)\hat{u} - \hat{u}_x\hat{u}
\end{equation}

Обозначим $\hat{\omega_i}(t) = \omega_i(t) - U'(i)$, $i = 0|1$. Тогда граничные условия для \eqref{fluct} запишутся как

\begin{gather}
  \hat{u}_x(0, t) = \hat{\omega}_0(t) \\*
  \hat{u}_x(1, t) = \hat{\omega}_1(t) \nonumber
\end{gather}


\subsubsection{Неустойчивость решения в open-loop и под radiation boundary feedback}

Мы изучим устойчивость происхождение (источник) open-loop системы \eqref{fluct}, линеаризуем систему, получаем 

\begin{gather} \label{linear_main}
  \theta_t = \theta_{xx} + 2\sigma(\tanh{(\sigma(x - \frac{1}{2})})\theta)_x \\* 
  \theta_x(0) = \theta_x(1) = 0
\end{gather}

где $\theta(x, t)$ линеаризация $\hat{u}$. Уравнение \eqref{linear_main} является уравнением диффузии-конвекции(адвекции)-реакции. Для простоты(облегчения) изучения устойчивости, мы ликвидируем (избавимся от) конвекционный (адвекционный) член, используя обратную (обратимую) преобразование $\zeta(x, t) = G(x)\theta(x, t)$, где

\begin{equation}
  G(x) = \frac{\cosh(\sigma(x - \frac{1}{2}))}{\cosh(\frac{\sigma}{2})}
\end{equation} 

которая из уравнения системы \eqref{linear_main} превратит ее в 

\begin{gather} \label{transf_linear}
  \zeta_t = \zeta_{xx} + \sigma^2(\frac{2}{\cosh^2(\sigma(x - \frac{1}{2}))} - 1)\zeta \\* 
  \zeta_x(0) = -\sigma\tanh(\frac{\sigma}{2})\zeta(0) \\* 
  \zeta_x(1) = \sigma\tanh(\frac{\sigma}{2})\zeta(1)
\end{gather}

Для $\sigma = 0$ система нейтрально (neutrally) устойчива. Для $\sigma > 0$, в добавок к граничным условиям будучи неустойчивым (anti-radiation) тип, реакционный член в \eqref{transf_linear} также является неустойчивым в окрестности $x = \frac{1}{2}$. Это показано на рисунке 2. При больших значения $\sigma > 0$, первое собственное значение становится более положительным. Для примера, при $\sigma = 15$, первое собственное значение - $0.6$.

Замечание: C radiation boundary feedback

\begin{equation}\label{radiation_feedback}
  \hat{\omega}_0(t) = k\hat{u}(0, t), \quad \hat{\omega}_1(t) = k\hat{u}(1, t), \quad k > 0,  
\end{equation} 

из источник [6], у системы \eqref{transf_linear} меняются только граничные условия

\begin{gather}\label{trans_new_cond}
  \zeta_x(0) = (k - \sigma\tanh(\frac{\sigma}{2}))\zeta(0) \\* 
  \zeta_x(1) = -(k - \sigma\tanh(\frac{\sigma}{2}))\zeta(1)
\end{gather}

Стабилизационные свойства системы \eqref{transf_linear} с новыми граничными условиями \eqref{trans_new_cond} улучшаются при $k \rightarrow +\infty$. Однако, численные значения собственных чисел показывают, что для $\sigma > \sigma*$, где $\sigma*$ примерно равно $10$, система всегда имеет точное неустойчивое собственное значение, так что не существует такого $k$, при котором система стабилизируется, показано на рис.3. Такие же отрицательные результаты подходят для нелинейных radiation boundary feedback, показаны в работе [13]. Таким образом, более изощренная форма обратной связи (feedback), либо является full-state feedback или dynamic feedback, которое необходимо, чтобы стабилизировать shock-like равновесие (даже локально)


\subsubsection{Full state feedback закон}

Мы выполним двух-шаговую feedback линеаризирующую конструкцию. На первом шаге, мы используем состояние (state) трансформации и feedback law (закон обратной связи) для $\omega_0$ для того чтобы линеаризовать преобразованное УМФ и его граничные условия при $x = 0$. На втором шаге, мы сконструируем закон обратной связи (feedback law) для $\omega_1$ чтобы стабилизировать результат линейной системы используя метод backstepping

Шаг А. Линейная трансформация и конструкция $\hat{\omega}_0$

Определим новую переменную состояние (state variable) $v(x, t)$ следующим образом:

\begin{equation}
  v(x, t) = \hat{u}(x, t)e^{-\frac{1}{2}\int\limits_0^x{[\hat{u}(y, t) + U(y)] dy}}
\end{equation}

которое может быть переписанно с помощью $G(x) = \frac{\cosh(\sigma(x - \frac{1}{2}))}{\cosh(\frac{\sigma}{2})}$, как

\begin{equation}\label{transf_w0}
  v(x, t) = G(x)\hat{u}(x, t)e^{-\frac{1}{2}\int\limits_0^x{\hat{u}(y, t) dy}}
\end{equation}


Замечание: Трансформация \eqref{transf_w0} является композицией преобразования Hopf-Cole на $\hat{u}(x, t)$ и трансформацией $d_x$ scaled by $G(x)$. Модель (шаблон) используется для изучения решений уравнения Бюргерса в работе [8]. Трансформация $d_x$ была использована в работе [22] для конструирования feedback control law для полулинейной параболической системы. The scaling $G(x)$
это стандартный критерий(размер) преобразования, используемый для удаления конвекционного члена в параболических УМФ.

Трансформация \eqref{transf_w0} из $\hat{u}$ в $v$ является обратимой

\begin{equation}
  \hat{u}(x) = \frac{v(x)/G(x)}{1 - \frac{1}{2}\int\limits_0^x{\frac{v(y)}{G(y)}dy}}
\end{equation}

Подставляя трансформацию \eqref{transf_w0} в систему \eqref{fluct}, мы получаем что $v$ удолетворяет следующим уравнениям

\begin{gather}
  v_t = v_{xx} - (U'(x) + \sigma^2)v + \frac{1}{2}(\hat{\omega}_0 - U(0)\hat{u}(0) - \frac{\hat{u}^2(0)}{2})v \\* 
  v_x(0) = \hat{\omega}_0 - \frac{1}{2}(\hat{u}(0) + U(0))\hat{u}(0) \\*
  v_x(1) = (\hat{\omega}_1 - \frac{1}{2}(\hat{u}(0) + U(0))\hat{u}(1)) \times e^{-\frac{1}{2}\int\limits_0^1{[\hat{u}(y, t) + U(y)] dy}}
\end{gather}

Устанавливая feedback law

\begin{equation}
  \hat{\omega}_0 = U(0)\hat{u}(0) + \frac{\hat{u}^2(0)}{2} = 2\sigma\tanh(\frac{\sigma}{2})\hat{u}(0) + \frac{\hat{u}^2(0)}{2}
\end{equation}

мы получаем следующую систему для $v$, в которой мы выражаем все коэф-ты явно, как это возможно

\begin{equation}\label{linear_rdpe}
  v_t = v_{xx} + \sigma^2(\frac{2}{\cosh^2(\sigma(x - \frac{1}{2}))} - 1)v
\end{equation}

с граничными условиями

\begin{gather} \label{need_stabilize}
  v_x(0) = \sigma\tanh(\frac{\sigma}{2})v(0) \\*
  v_x(1) = \sigma\tanh(\frac{\sigma}{2})v(1) + (\hat{\omega}_1 - \frac{\hat{u}(1)^2}{2}) \times (1 - \frac{1}{2}\int\limits_0^1{\frac{\cosh(\sigma/2)v(y)}{\cosh(\sigma(y - 1/2))} dy})
\end{gather}

Заметим, что \eqref{linear_rdpe} линейное параболическое уравнение реакции-диффузии с тем же самым дестабилизирующим коэф-том что и в системе \eqref{transf_linear}



Шаг Б. Конструкция $\hat{\omega}_1$ и использование метода backstepping

Для того, чтобы найти feedback law для $\hat{\omega}_1$ который стабилизирует \eqref{need_stabilize}, мы используем метод backstepping для одномерных параболических уравнений (представлено в работе [19]). Мы определим новое состояние $w(x, t)$, которое получается из $v$ преобразованием

\begin{equation}
  w(x, t) = v(x, t) - \int\limits_0^x{k(x, y)v(y, t)dy}
\end{equation}

Состояние $w$ удолетворяет следующему УМФ:

\begin{gather}\label{target_system}
  w_t = w_{xx} + \sigma^2(\frac{1}{\cosh^2(\sigma(x - \frac{1}{2}))} - 1)w - cw\\*
  w_x(0) = \sigma\tanh(\frac{\sigma}{2})w(0) \\*
  w_x(1) = -\sigma\tanh(\frac{\sigma}{2})w(1)
\end{gather}

где $c \ge 0$ параметр управления. Система \eqref{target_system} будет упоминаться (называться) целевой системой.

Замечание 4.2: Реакционный член в \eqref{target_system} неположительный, так что $w$-система экспонентциально устойчива. Заметим, что мы неполностью избавились от реакционого члена в исходной (первоисходной) системе \eqref{need_stabilize}, но мы только уменьшили его, чтобы устранить его положительную часть, не тратя усилия управления на изменение его отрицательной части см. рис.4.

Замечание 4.3: Коэф-т $c$ может быть нулевым когда $\sigma > 0$. Когда $\sigma = 0$, мы должны выбрать $c > 0$, потому что результатирующая система $w_t = w_{xx} - cw$, $w_x(0, t) = w_x(1, t) = 0$ может быть устойчива только при $c = 0$.

Мы должны определить ядро $k(x, y)$, так чтобы $w$ удолетворял \eqref{target_system}. Следуя работы [19], мы находим что ядро $k$ удолетворяет следующим равенствам:

\begin{gather}
  k_{xx} = k_{yy} + \sigma^2(1 - 2\tanh^2{(\sigma(y - 1/2))} + \tanh^2{(\sigma(x - 1/2))})k + ck\\*
  k(x, x) = -\sigma/2(\tanh{(\sigma(x - 1/2))} + \tanh{(\sigma/2)}) - cx/2\\*
  k_y(x, 0) = \sigma\tanh{(\sigma/2)}k(x, 0)
\end{gather}

которое является линейным гиперолическим УМФ в области 

\begin{equation}
  \tau = \left\{(x, y): 0 \le y \le x \le 1 \right\}
\end{equation}

В работе [19], показано, что $k \in C^2(\tau)$. Ядро $k$ может быть вычислено численно или символьно [19]

Из определения $w_x(1)$, $w(x)$ и $v_x(1)$ мы находим, что управление (control law)

\begin{gather}
  \hat{\omega}_1(t) = \frac{\hat{u}^2(1, t)}{2} + (k(1, 1) - 2\sigma\tanh{(\sigma/2)})\hat{u}(1, t) \\* + 
  \int\limits_0^1{(k_x(1, y) + \sigma\tanh{\sigma/2}k(1, y)} \\ \nonumber 
  \times G(y)e^{(\int\limits_y^1{\hat{u}(\xi, t)d\xi}\hat{u}(y, t) dy)} \nonumber
\end{gather}

\subsubsection{Стабилизация замкнутой (closed-loop) системы под full-state control law}

Обозначим начальное условие как $\hat{u}_0 = \hat{u}(x, 0)$ и определим класс $K_{\infty}$ [12] функций
\begin{equation}
  g(r) = \frac{r}{2}e^{\frac{r}{2}}
\end{equation}

Теорема 1: Предположим что $\hat{u}_0 \in H^2$ такое что 

\begin{equation}
  \norm{\hat{u}_0}_{L^2} < g^{-1}(\frac{1}{m}\sqrt{\frac{\sigma}{\sinh{h\sigma}}})
\end{equation}

и что оно удовлетворяет условиям совместимости


\begin{gather}
  \hat{u}'_0(0) = 2\sigma\tanh({\sigma/2})\hat{u}_0(0) + \hat{u}^2_0(0)/2\\*
  \hat{u}'_1(1) = \hat{u}^2_0(1)/2 + (k(1, 1) - 2\sigma\tanh{(\sigma/2)}) \hat{u}_0(1) \\ \nonumber
  + \int\limits_0^1{(k_x(1, y) + \sigma\tanh{(\sigma/2)}k(1, y)) \times G(y)e^{\int\limits_0^1{\hat{u}_0(\xi)d\xi}}\hat{u}_0(y)dy}
\end{gather}

Тогда равновесный profile $\hat{u} \equiv 0$ системы \eqref{fluct} с законами обратной связи (feedback laws) $\hat{\omega}_0(t)$ и $\hat{\omega}_1(t)$ является экспонентциально устойчивой в норме $L^2$, т.е. $\forall t > 0$,

\begin{equation}
  \norm{\hat{u}(t)} \le \frac{\sqrt{2\sigma\coth{(\sigma/2)}}g(\norm{\hat{u}_0}_{L^2})}{\frac{1}{m}\sqrt{\sigma/\sinh{(h\sigma)}} - g(\norm{\hat{u}_0}_{L^2})}e^{-\alpha t}   
\end{equation} 

где $m, \alpha > 0$

Замечание 5.1: Теорема 1 это утверждение областной (меньше чем глобальной, больше чем бесконечномалой локальной) устойчивости в пространстве $L^2$

\subsubsection{Моделирование}

А. open-loop система

Система open-loop неустойчива, как показано в разделе 3. Численное изучение линеаризованной системы вокруг shock profile, показывает наличие одного положительного (хотя, возможно, небольшого) собственного значения для любого $\sigma > 0$. На рис. 5 можно видеть конечно-временное раздутие (blow up) open-loop системы для $\sigma = 3$.

Б. Radiation boundary feedback

В замечании 3.1 мы обьяснили что shock profiles является неустойчивыми (даже локально) radiation feedback для $\sigma > \sigma*$, где $\sigma \approx 10$. Отсутствие стабилизируемости по обратной связи излучения (radiation feedback) в соответствии с его целью быть просто регулирование скалярных выходов $\hat{u}(0, t)$ и $\hat{u}(1, t)$ к нулю, а не стабилизации всего состояние $\hat{u}(x, t)$. Для малого $\sigma$, последнее может быть достигнуто в качестве бонуса, следствие устойчивости основной нулевой динамики (zero dynamics). Тем не менее, для больших $\sigma$ zero dynamics стала неустойчивой, и, таким образом, регулирование output не сопровождается устойчивость замкнутой (closed-loop) системы.

Несмотря на неустойчивость линеаризованного уравнения Бюргерса с radiation feedback на shock profiles, не так все плохо с с полным нелинейным моделированияем. Нелинейности предотвращают фантастический конечное временное раздутие (finite time blow-up) (а не заставляя его), и решения, по замкнутой системы (closed-loop) просто сходятся к другому равновесию (к не желаемому shock profile равновесию). На фиг. 6 можно увидеть сходимость $u$ к двум различным несимметричных равновесиям, которые, кажется, (почти) соответствует желаемой shock profile только на границах, но нигде больше.

В моделировании мы узнали, что для нечетных начальных условий в окрестности $х = 1/2$ radiation feedback \eqref{radiation_feedback} достигает сходимость к $U(x)$. Тем не менее, для несимметричных начальных условий, система переходит (стремится) к другим стационарным решениям.

С. Backstepping feedback

Перед тем, как показать результаты с нелинейной конструкцией, мы покажем результаты линеаризованного backstepping управления


\begin{gather}\label{cl_controller}
  \hat{\omega}'_0(t) = 2\sigma\tanh({\sigma/2})\hat{u}(0, t)\\*
  \hat{\omega}'_1(t) = -(3\sigma\tanh{(\sigma/2)} + c/2)\hat{u}(1, t) - \int\limits_0^1{p(y)G(y)\hat{u}(y, t)dy}
\end{gather}


Ядро $-p(x)$ и решение замкнутой системы с этими управлениями показаны на рис.7. Radiation feedback не стабилизирует нужное равновесие profile в этом случае, как показано на рис. 6.

На рис.8. мы показываем решения при различных начальных начальных условиях, под воздействием нелинейного управления backstepping

\subsubsection{Заключение}

Мы решили проблему full-state стабилизацией для полного нелинейного вязкого уравнения Бюргерса, где мы решали сконструировали наблюдателя, output feedback стабилизацию, генерацию траектории (trajectory generation, and tracking). Наши результаты основываются на нелинейном feedback линеаризующим преобразовании, которое позволяет нам использовать линейный конструктор управления метод backstepping. Из-за явного характера преобразовании и наших методов, мы смогли получить формулы для законов обратной связи. Так как наша трансформация не является глобально обратимой, наши результаты не являются глобальными, что согласуется с отсутствием глобальной управляемости, показано в [10], [11], однако мы получили не бесконечно малый область притяжения (region of attraction), используя высокоуровневый набор функции Ляпунова в областях целесообразности закона управления и обратимости преобразования. 

Интересные проблемы будущих исследований, которые включают в себя конструкцию, которая использует только одно входное управление, $\hat{\omega}_1$ (с $\hat{\omega}_0$ равным нулю), расширение на более общие нелинейные параболические уравнения в частных производных с конвективными нелинейности и конвективных нелинейности более общем виде, в уравнений Бюргерса в высших измерениях, и в более сложных уравнений в частных производных с конвективными нелинейностей, например Курамото-Сивашинского и Навье-Стокса.